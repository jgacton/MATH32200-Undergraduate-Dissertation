%\documentclass[handout]{beamer}
\documentclass[t]{beamer} % [t] aligns itemize/enumerate lists at the top
\usetheme{default}%{Boadilla}%{Singapore}%{Boadilla}%{Goettingen}%{Pittsburgh}
\usecolortheme{orchid}
\setbeamertemplate{navigation symbols}{}
\usepackage{hyperref}
\hypersetup{colorlinks=true,urlcolor=blue,citecolor=blue,linkcolor=blue}

\title{Some tips on \\ preparing and giving a project presentation\\
{\small University of Bristol}}
\author{Andrew Booker\\ \texttt{andrew.booker@bristol.ac.uk}}
\date{22nd February 2024}

\begin{document}
\begin{frame}
\titlepage
\end{frame}

\begin{frame}
\frametitle{What I will present today}
\begin{itemize}
\item Advice on how to \emph{prepare}, \emph{write}, and \emph{deliver} a mathematics talk in general, and a project presentation in particular 
\item What constitutes a good talk depends a lot on your taste, as well
as the field, but there are some basics that are universal
\item When in doubt, ask your supervisor
\end{itemize}
\end{frame}

\begin{frame}
\frametitle{What is expected of you}
\begin{itemize}
\item 20 cp projects: Give a {\textcolor{red}{15-minute}}  presentation on your project---this should {\textcolor{red}{include 2-3 minutes time for questions}}
\item 40 cp project: Give a {\textcolor{red}{20-minute}} presentation on your project plus an {\textcolor{red}{additional 5 minutes for questions}}
\pause
\item Make it understandable, not too technical
\item Can give an overview or concentrate on the most important/interesting part
\end{itemize}
\end{frame}

\begin{frame}
\frametitle{What is expected of you}
\includegraphics[scale=0.4]{talkmark.pdf}
\end{frame}

\begin{frame}
\frametitle{Preparing your talk}
How to start
\begin{itemize}
\item Make sure you know the following:
\begin{itemize}
\item What do you want to convey? 
\item Who is your audience? (Experts? Fellow students?)
\item How much time do you have?
\item How will you deliver the talk? (slides, whiteboard, \ldots)
\item What is the venue like? 
\end{itemize}

\pause

\item Make an outline
\begin{itemize}
\item Introduce your talk. What will you tell us?  Why should we care? 
\item Main body---divide into sections if appropriate. Tell a story!
\item Concluding remarks.
\end{itemize}
\end{itemize}
\end{frame}

\begin{frame}
\frametitle{Preparing your talk}

Some things to think about:
\begin{itemize}
\item Don't include every detail you have learned!
\item Simplify. Leave out details that are too technical (you can mention that you do)
\item Concentrate on the \emph{story}. What you spent the most time working on is not necessarily what helps the story the most. 
\begin{itemize}
\item For pure maths, this might include some background on the history
of the problem and why people got interested in it.
\item For applied, talk about how your project (or the general area it
lies in) relates to everyday life.
\end{itemize}
\pause
\item Think about good presentations/lectures you have seen
\item Think about the bad ones too :-)
\item How would you explain your project to a friend? How would you explain it to your supervisor? 
\item If necessary: make at least the first 1/3 of your talk accessible
to everyone in the audience, and the rest to experts (but say that this is what you do!)
\end{itemize}
\end{frame}

\begin{frame}
\frametitle{Writing the talk}

\begin{itemize}
\item You can use slides (projector) or write on the board. Since the talk is short, using slides of often the safer/better choice. 
\begin{itemize}
\item Use \emph{Beamer} to produce typed slides (see the
\href{https://www.overleaf.com/learn/latex/Beamer\_Presentations\%3A\_A\_Tutorial\_for\_Beginners\_(Part\_1)\%E2\%80\%94Getting\_Started}{tutorial on Overleaf},
or the source for these slides on Blackboard)
\item Handwritten (live or pre-written) slides
\end{itemize}
\pause
\item {\textcolor{red}{DON'T}}
\begin{itemize}
\item Do not fill your slides (especially with formulas!) 
\item Do not write everything you plan to say 
\item Do not use too much notation and specialised terminology
\end{itemize}
\pause
\item {\textcolor{green}{DO}}
\begin{itemize}
\item Write enough so the audience can follow
\item Use pictures if appropriate. Hand drawn are perfectly fine! 
\end{itemize}
\end{itemize}
\end{frame}

\begin{frame}
\frametitle{Actually writing the talk}
\begin{enumerate}
\item Write a short part of your talk 
\item Practise it (and time it)
\item Write the rest
\item Practise it (and time it), preferably in front of someone
\item Proof read, make tweaks
\item Practise, practise, practise! 
\end{enumerate}
\end{frame}

\begin{frame}
\frametitle{Writing the talk}
\begin{itemize}
\item More things to think about: 
\begin{itemize}
\item Most common criticism: too technical or went over time  
\item You are an expert on the topic, your audience is not!
\item Build in flexibility
\begin{itemize}
\item Have an exit strategy
\item Finish on time! (But not after half the time!) 
\end{itemize}
\item You will probably speak faster during the actual talk
\item Giving talk online: get used to your software. (Can you use a pointer?)
\end{itemize}
\end{itemize}
\end{frame}

\begin{frame}
\frametitle{Delivering the talk}
\begin{itemize}
\item Slow down! 
\item Look at your audience
\item Don't read, don't memorise (practise---but not verbatim) 
\item Motivate your topic, explain why \emph{you} find it interesting 
\item Don't assume too much of your audience. Explain your terms, repeat and recall things throughout. (We're bad at paying attention!) 
\item You are not expected to know everything\ldots
\begin{itemize}
\item \ldots but don't pretend to know things you don't  
\item Instead, if you say things you don't really understand, say so.
\item It is OK to say ``I don't understand exactly how, but I have read
that this has connections to\ldots'' or ``My supervisor told me that\ldots''
\end{itemize}
\end{itemize}
\end{frame}

\begin{frame}
\frametitle{Delivering the talk}
\begin{itemize}
\item Answering questions
\begin{itemize}
\item It is OK to pause and think.
\item It is OK not to know an answer. 
\begin{itemize}
\item Just show some interest: \\ ``Interesting question, I haven't thought about that...";\\
``That was outside the scope of my project..."; \\``That's something I would definitely like to learn more about..."
\item OK to say that you need to think about it more, ask if you can talk afterwards
\end{itemize}
\end{itemize}
\end{itemize}
\end{frame}

\begin{frame}
\frametitle{Remember}
\begin{itemize}
\item Have fun! 
\item Purpose of project talk: 
\begin{itemize}
\item 1. Provide nice conclusion to your project where you get to show what you have learned
\item 2. Get some practice in giving presentations
\end{itemize}
\item Worth only 10\% of your total mark for the project
\end{itemize}
\end{frame}

\begin{frame}
\frametitle{Final tips}
\begin{itemize}
\pause\item{{\textcolor{red}{Don't say:}}}
\begin{itemize}
\item Consider the---up to multiplicative constant---unique $\mathbb{Z}^2$-invariant locally finite Radon measure on the Cartesian product of two copies of the Hausdorff, locally compact space of real numbers whose topology is generated by intervals, equipped with the product topology \end{itemize}
\pause\item {\textcolor{green}{When you could say:}}
\begin{itemize}
\item Consider the usual notion of area in the plane $\mathbb{R}^2$. 
\end{itemize}

\pause
\item Don't overuse the \texttt{\textbackslash pause} command in Beamer.
\end{itemize}
\end{frame}

\begin{frame}
\frametitle{}
\begin{center}
\large{Thank you for your attention!}\\
\vspace{1cm}
Questions?
\vspace{2cm}

email: \texttt{andrew.booker@bristol.ac.uk}
\end{center}
\end{frame}
\end{document}
