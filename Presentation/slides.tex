\documentclass{beamer} % [t] aligns itemize/enumerate lists at the top
\usetheme{Boadilla}
\usecolortheme{default}

\title{Automated Market-Making}
\author{Joshua Acton}
\date{22nd March 2024}
\subtitle{Supervised by Professor Nick Whiteley}

% Title slide
\begin{document}
\begin{frame}
    \titlepage
\end{frame}

% Introduction slide
\begin{frame}{Introduction}
    \begin{itemize}
    \item How do financial markets work? (at the level of individual participants)
    \item The Avellaneda-Stoikov Model
    \item Results on the statistical properties of the limit orderbook
    \end{itemize}
\end{frame}

\begin{frame}{What is a (financial) market?}
    \begin{itemize}
        \item Stocks, bonds, commodities, all manner of derivatives
        \item Multiple buyers, multiple sellers
        \item Prices determined by trading: literally supply \& demand
        \item So how does trading actually occur inside the exchange?
    \end{itemize}
\end{frame}

\begin{frame}{The limit orderbook}
    \begin{center}
        \begin{tabular}{ |c|c|c| } 
            \hline
            Side & Price /£ & Volume \\ 
            \hline
            A & 1.02 & 50 \\
            A & 1.01 & 30 \\
            \_ & 1.00 & 0 \\
            B & 0.99 & 25 \\ 
            B & 0.98 & 45 \\
            \hline
        \end{tabular}
    \end{center}
    \begin{itemize}
        \item \emph{Limit} orders guaruntee price but not execution
        \begin{itemize}
            \item Limit orders can also be ammended/updated for as long as they exist
        \end{itemize}
        \item \emph{Market} orders guaruntee execution but not price
        \item \emph{Spread} between best bid and best ask $\rightarrow$ £0.02
        \item Can view the mid-price (here £1.00) as the ``true'' price
    \end{itemize}
\end{frame}

\begin{frame}{The limit orderbook}
    After market order for 20 shares:
    \begin{center}
        \begin{tabular}{ |c|c|c| } 
            \hline
            Side & Price /£ & Volume \\ 
            \hline
            A & 1.02 & 50 \\
            A & 1.01 & \textcolor{red}{10} \\
            \_ & 1.00 & 0 \\
            B & 0.99 & 25 \\ 
            B & 0.98 & 45 \\
            \hline
        \end{tabular}
    \end{center}
    After another market order for 30 shares:
    \begin{center}
        \begin{tabular}{ |c|c|c| } 
            \hline
            Side & Price /£ & Volume \\ 
            \hline
            A & 1.02 & \textcolor{red}{30} \\
            A & 1.01 & \textcolor{red}{0} \\
            \_ & 1.00 & 0 \\
            B & 0.99 & 25 \\ 
            B & 0.98 & 45 \\
            \hline
        \end{tabular}
    \end{center}
    So who is doing the trading?
\end{frame}

\begin{frame}{Market participants}
    \begin{itemize}
        \item Investors
        \begin{itemize}
            \item Pension funds, asset managers, governments, some hedge funds
        \end{itemize}
        \item Speculators
        \begin{itemize}
            \item Other hedge funds, proprietary trading firms
        \end{itemize}
        But what if no one wants to sell? (resp. buy?)

        What if buyers and sellers have wildly different indifference prices?
        \item Dealers
    \end{itemize}
\end{frame}

\begin{frame}{Dealer considerations}
    Idea:
    \begin{itemize}
        \item Simultaneously place bid and ask limit orders $\rightarrow$ simultaneously buying and selling
        \item Enables other market participants to always have someone to trade against $\rightarrow$ ``providing liquidity''
        \item Narrows the spread between bid and ask prices, decreasing implied cost of trading
        \item Dealer profits the (small) spread between buying and selling (multiplied across large trading volume)
    \end{itemize}
    Risks:
    \begin{itemize}
        \item Informed traders
        \item Inventory
    \end{itemize}
    Who does this?
    \begin{itemize}
        \item Specialist HFT/MM firms
        \item Investment banks
    \end{itemize}
\end{frame}

\begin{frame}{Modelling dealer behavior}
    If dealer accrues positive inventory:
    \begin{itemize}
        \item Want to sell more than buy
        \item Set lower ask price
    \end{itemize}
    If dealer accrues negative inventory:
    \begin{itemize}
        \item Want to buy more than sell
        \item Set a higher bid price
    \end{itemize}
    Other potential considerations:
    \begin{itemize}
        \item If high price volatility, set a wider spread 
        \item If trading day ends sooner, set narrower spread
        \item Dealer may also have some predetermined risk aversion parameter
    \end{itemize}
    Now for the maths\dots
\end{frame}

\begin{frame}{The Avellaneda-Stoikov model}
    Model market mid-price as Brownian motion with variance $\sigma^2$ (no drift)
    $$\mathrm d S_t=\sigma \mathrm d W_t,\;t\in[0,T]$$
    Dealer's value function: Expected exponential utility of terminal wealth
    \begin{equation}\label{1}
        v(x,s,q,t)=\mathbb{E}\left[-e^{-\gamma(x+qS_T)}|\mathcal{F}_t\right]
    \end{equation}
    \begin{itemize}
        \item $x=$ dealer's initial wealth (cash)
        \item $q=$ dealer's inventory (assume fixed for now)
        \item $\gamma=$ dealer's risk aversion
    \end{itemize}
\end{frame}

\begin{frame}{The Avellaneda-Stoikov model}
    We can find a reservation bid price: The trading price at which the dealer is indifferent between buying an extra share and doing nothing: Set
    \begin{equation}\label{2}
        v(x-r^b(s,q,t),s,q+1,t)=v(x,s,q,t)
    \end{equation}
    and by substitution of (\ref{1}) into (\ref{2}) we obtain
    \begin{equation}\label{3}
        r^b(s,q,t)=s+(-1-2q)\frac{\gamma\sigma^2(T-t)}{2}
    \end{equation}
    An analagous expression exists for $r^a.$ We define the dealer's reservation price $r(s,q,t):=\frac{r^a(s,q,t)+r^b(s,q,t)}{2}$ and obtain
    \begin{equation}\label{4}
        r(s,q,t)=s-q\gamma\sigma^2(T-t)
    \end{equation}
\end{frame}

\begin{frame}{The Avellaneda-Stoikov model}
    Now consider a dealer who sets limit orders. 
    The dealer quotes bid $p^b$ and ask $p^a$ with spreads $\delta^b=s-p^b$ and $\delta^a=p^a-s$
    respectively.
    We assume that 
    \begin{itemize}
        \item Market buy orders will `lift' the dealer's sell orders at Poisson rate $\lambda^a(\delta^a)$ (a decreasing function of $\delta^a$)
        \item Market sell orders will `hit' the dealer's bid orders at rate $\lambda^b(\delta^b)$ (decreasing in $\delta^b$).
    \end{itemize}
    Now have stochastic wealth and inventory:
    \begin{equation}
        \mathrm dX_t=p^a\mathrm dN_t^a-p^b\mathrm dN_t^b
    \end{equation}
    \begin{equation}
        q_t=N_t^b-N_t^a
    \end{equation}
\end{frame}

\begin{frame}{The Avellaneda-Stoikov model}
    Need to adapt our objective function: Maximise terminal wealth over possible bid/ask spreads $\delta^a,\delta^b$
    \begin{equation}
        u(s,x,q,t)=\max_{\delta^a,\delta^b}\mathbb{E}\left[-e^{-\gamma(X_T+q_TS_T)}|\mathcal{F}_t\right]
    \end{equation}
    We now have a \emph{stochastic optimal control} problem:
    \begin{itemize}
        \item Formulate \emph{Hamilton-Jacobi-Bellman} equation and solve for function $u$
        \item Use function $u$ to determine optimal $\delta^a,\delta^b$
    \end{itemize}
    Computationally difficult, however through some asymptotic approximations we can 
    work out some simple expressions for an approximate solution in terms of our model paramaters\dots
\end{frame}

\begin{frame}{The Avellaneda-Stoikov model}
    We obtain
    \begin{equation}
        r(s,q,t)=s-q\gamma\sigma^2(T-t)
    \end{equation}
    which coincides with our indifference price for the dealer with static inventory, and
    \begin{equation}
        \delta^a+\delta^b=\gamma\sigma^2(T-t)+\frac{2}{\gamma}\log\left(1+\frac{\gamma}{k}\right)
    \end{equation}
    with $\gamma$ and $\sigma$ as before, $k$ is a parameter from the orderbook describing how 
    market order size impacts prices.
\end{frame}

\begin{frame}{Statistical properties of the limit order book}
    Poisson intensity $\lambda$ describes how likely a limit order is to be executed as a function of it's 
    distance $\delta$ to the mid-price. Need some statistics regarding:
    \begin{itemize}
        \item Overall frequency of market orders
        \begin{itemize}
            \item For simplicity, assume constant $\Lambda$
        \end{itemize}
        \item Size distribution of market orders
        \begin{itemize}
            \item ``Econophysics'' $\implies$ power law $f^Q(x)\propto x^{-1-\alpha}$
        \end{itemize}
        \item Price impact of large market orders
        \begin{itemize}
            \item ``Econophysics'' $\implies$ either $\Delta p\propto Q^\beta$ or $\Delta p\propto\log Q$
        \end{itemize}
    \end{itemize}
    Using the first result for price impact we obtain:
    \begin{align*}
        \lambda(\delta)=\Lambda\mathbb{P}(\Delta p>\delta)&=\Lambda\mathbb{P}(\log Q>K\delta)\\
        &=\Lambda\mathbb{P}\left(Q>e^{K\delta}\right)\\
        &=\Lambda\int_{e^{K\delta}}^{\infty}x^{-1-\alpha}dx\\
        &=Ae^{-k\alpha\delta}
    \end{align*}
\end{frame}

\begin{frame}{The Avellaneda-Stoikov model - Summary}
    At each timestep, given current inventory and parameters estimated from order book:
    \begin{itemize}
        \item Compute reservation price $r(s,q,t)$
        \item Compute spread $\delta^a+\delta^b$
        \item Set quotes $p^a=s+\frac{\delta^a+\delta^b}{2}$, $p^b=s-\frac{\delta^a+\delta^b}{2}$
    \end{itemize}
\end{frame}

% Conclusion
\begin{frame}{Conclusion}
    \begin{itemize}
        \item Through the framework of stochastic control, we can attempt to model the optimal behaviour of a dealer in financial markets
        \item We can also consider: 
        \begin{itemize}
            \item Geometric Brownian Motion
            \item Infinite time horizons
            \item Informed trader risk (game theory)
            \item Alternative models for market orders (Hawkes Processes)
        \end{itemize}
    \end{itemize}
\end{frame}

% Thank you & questions slide
\begin{frame}
    \frametitle{}
    \begin{center}
        \large{Thank you for your attention!}\\
        \vspace{1cm}
        Questions?
        \vspace{2cm}

        email: \texttt{josh.acton.2021@bristol.ac.uk}
    \end{center}
\end{frame}

\end{document}
