\documentclass{beamer} % [t] aligns itemize/enumerate lists at the top
\usetheme{Boadilla}
\usecolortheme{default}

\title{Automated Market-Making}
\author{Joshua Acton}
\date{18th March 2024}
\subtitle{Supervised by Professor Nick Whiteley}

% Title slide
\begin{document}
\begin{frame}
    \titlepage
\end{frame}

% Introduction slide
\begin{frame}{Introduction}
    \begin{itemize}
    \item How do financial markets work? (at the level of individual participants)
    \item The Avellaneda-Stoikov Model
    \item Results on the statistical properties of the limit orderbook
    \end{itemize}
\end{frame}

\begin{frame}{What is a market?}
    \begin{itemize}
        \item Multiple buyers, multiple sellers
        \item Prices determined by trading: literally supply \& demand
        \item So how does trading actually occur inside the exchange?
    \end{itemize}
\end{frame}

\begin{frame}{The limit orderbook}
    \begin{center}
        \begin{tabular}{ |c|c|c| } 
            \hline
            Side & Price /\$ & Volume \\ 
            \hline
            A & 1.02 & 50 \\
            A & 1.01 & 30 \\
            \_ & 1.00 & 0 \\
            B & 0.99 & 25 \\ 
            B & 0.98 & 45 \\
            \hline
        \end{tabular}
    \end{center}
    \begin{itemize}
        \item \emph{Limit} orders guaruntee price but not execution
        \begin{itemize}
            \item Limit orders can also be ammended/updated for as long as they exist
        \end{itemize}
        \item \emph{Market} orders guaruntee execution but not price
        \item From a CS perspective, the order book is a \emph{priority queue}
    \end{itemize}
    So who is doing the trading?
\end{frame}

\begin{frame}{Market participants}
    \begin{itemize}
        \item Investors
        \item Speculators
        \begin{itemize}
            \item What if no one wants to sell? (resp. buy?)
            \item What if buyers and sellers have wildly different indifference prices?
        \end{itemize}
        \item Dealers
    \end{itemize}
\end{frame}

\begin{frame}{Dealer considerations}
    Idea:
    \begin{itemize}
        \item Simultaneously place bid and ask limit orders $\rightarrow$ simultaneously buying and selling
        \item Enables other market participants to always have someone to trade against
        \item Narrows the spread between bid and ask prices, increasing efficiency
        \item Dealer profits the (small) spread between buying and selling (multiplied by a large trading volume)
    \end{itemize}
    Risks:
    \begin{itemize}
        \item Informed traders
        \item Inventory
    \end{itemize}
\end{frame}

\begin{frame}{Modelling dealer behavior}
    \begin{itemize}
        \item If large positive inventory, set a lower ask price
        \item If large negative inventory, set a higher bid price
        \item If high price volatility, set a wider spread 
        \item Dealer may also have some predetermined risk aversion parameter
    \end{itemize}
\end{frame}

\begin{frame}{The Avellaneda-Stoikov model}
    Two-step process:
    \begin{itemize}
        \item Given current inventory (\& other fixed params), compute indifference price for the asset
        \item Calibrate bid and ask quotes to the current order book
        \begin{itemize}
            \item Consider probability of order execution as a function of distance from mid-price
        \end{itemize}
    \end{itemize}
\end{frame}

\begin{frame}{Statistical properties of the limit order book}
    \begin{itemize}
        \item Frequency of market orders: Poisson process (fixed rate)
        \item Size distribution of market orders: Power law. Exponent is market-specific
        \item Price impact of large market orders: Logarithmic or square root
    \end{itemize}
\end{frame}

% Conclusion
\begin{frame}{Conclusion}
    \begin{itemize}
        \item Under a few (potentially unrealistic) assumptions, it is
        possible to determine what the optimal dealer strategy should be
        \item No one model for all markets
        \item Still very much an open field of study bringing together ideas
        from maths, stats, physics, computer science and economics
    \end{itemize}
\end{frame}

% Thank you & questions slide
\begin{frame}
    \frametitle{}
    \begin{center}
        \large{Thank you for your attention!}\\
        \vspace{1cm}
        Questions?
        \vspace{2cm}

        email: \texttt{josh.acton.2021@bristol.ac.uk}
    \end{center}
\end{frame}

\end{document}
