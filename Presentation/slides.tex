\documentclass{beamer} % [t] aligns itemize/enumerate lists at the top
\usetheme{Boadilla}
\usecolortheme{default}

\title{Automated Market-Making}
\author{Joshua Acton}
\date{22nd March 2024}
\subtitle{Supervised by Professor Nick Whiteley}

% Title slide
\begin{document}
\begin{frame}
    \titlepage
\end{frame}

% Introduction slide
\begin{frame}{Introduction}
    \begin{itemize}
    \item How do financial markets work? (at the level of individual participants)
    \item How can we use stochastic control theory to model the behaviour of market participants?
    \item The Avellaneda-Stoikov Model
    \item Results on the statistical properties of the limit orderbook
    \end{itemize}
\end{frame}

\begin{frame}{What is a (financial) market?}
    \begin{itemize}
        \item Stocks, bonds, commodities, all manner of derivatives
        \item Multiple buyers, multiple sellers
        \item Prices determined by trading: literally supply \& demand
        \item So how does trading actually occur inside the exchange?
    \end{itemize}
\end{frame}

\begin{frame}{The limit orderbook}
    \begin{itemize}
        \item Can either place \emph{limit} orders or \emph{market} orders
        \item \emph{Limit} orders guaruntee price but not execution
        \begin{itemize}
            \item Limit orders can also be ammended/updated for as long as they exist
        \end{itemize}
        \item \emph{Market} orders guaruntee execution but not price
        \item \emph{Spread} between best bid and best ask $\rightarrow$ £0.02
        \item Can view the mid-price (here £1.00) as the ``true'' price
    \end{itemize}
    \begin{center}
        \begin{tabular}{ |c|c|c| } 
            \hline
            Side & Price /£ & Volume \\ 
            \hline
            A & 1.02 & 50 \\
            A & 1.01 & 30 \\
            \_ & 1.00 & 0 \\
            B & 0.99 & 25 \\ 
            B & 0.98 & 45 \\
            \hline
        \end{tabular}
    \end{center}
\end{frame}

\begin{frame}{The limit orderbook}
    After market buy order for 20 shares:
    \begin{center}
        \begin{tabular}{ |c|c|c| } 
            \hline
            Side & Price /£ & Volume \\ 
            \hline
            A & 1.02 & 50 \\
            A & 1.01 & \textcolor{red}{10} \\
            \_ & 1.00 & 0 \\
            B & 0.99 & 25 \\ 
            B & 0.98 & 45 \\
            \hline
        \end{tabular}
    \end{center}
    After another market buy order for 30 shares:
    \begin{center}
        \begin{tabular}{ |c|c|c| } 
            \hline
            Side & Price /£ & Volume \\ 
            \hline
            A & 1.02 & \textcolor{red}{30} \\
            A & 1.01 & \textcolor{red}{0} \\
            \_ & 1.00 & 0 \\
            B & 0.99 & 25 \\ 
            B & 0.98 & 45 \\
            \hline
        \end{tabular}
    \end{center}
\end{frame}

\begin{frame}{Dealer considerations}
    Problem:
    \begin{itemize}
        \item What if no one wants to sell? (resp. buy?)
        \item What if buyers and sellers have wildly different price expectations?
    \end{itemize}
    Idea:
    \begin{itemize}
        \item Simultaneously place bid and ask limit orders $\rightarrow$ simultaneously buying and selling
        \item Enables other market participants to always have someone to trade against $\rightarrow$ ``providing liquidity''
        \item Narrows the spread between bid and ask prices, decreasing implied cost of trading
        \item Dealer profits the (small) spread between buying and selling (multiplied across large trading volume)
    \end{itemize}
    Risks:
    \begin{itemize}
        \item Informed traders
        \item Inventory
    \end{itemize}
\end{frame}

\begin{frame}{Modelling dealer behavior}
    If dealer accrues positive inventory:
    \begin{itemize}
        \item Want to sell more than buy
        \item Set lower ask price
    \end{itemize}
    If dealer accrues negative inventory:
    \begin{itemize}
        \item Want to buy more than sell
        \item Set a higher bid price
    \end{itemize}
    Now the maths\dots
\end{frame}

\begin{frame}{Stochastic Optimal Control}
    Following Pham (2009):

    System:
    \begin{equation}
        \mathrm dX_t=b(X_t,\alpha_t)\mathrm dt+\sigma(X_t,\alpha_t)\mathrm dW_t
    \end{equation}
    Objective (Finite time horizon):
    \begin{equation}
        v(t,x):=\sup_{\alpha\in\mathcal A}\mathbb{E}\left[\int_t^Tf(s,X_s^{t,x},\alpha_s)\mathrm ds+g(X_T^{t,x})\right]
    \end{equation}
    Dynamic Programming Principle:
    \begin{equation}
        v(t,x)=\sup_{\alpha\in\mathcal A}\mathbb{E}\left[\int_t^\theta f(s,X_s^{t,x},\alpha_s)\mathrm ds+v(\theta,X_\theta^{t,x})\right]
    \end{equation}
    Hamilton-Jacobi-Bellman Equation:
    \begin{equation}
        \frac{\partial v}{\partial t}(t,x)+\sup_{\alpha\in\mathcal{A}}\left[\mathcal{L}^\alpha v(t,x)+f(t,x,\alpha)\right]=0
    \end{equation}
\end{frame}

\begin{frame}{The Avellaneda-Stoikov model}
    Following Avellaneda \& Stoikov (2008):

    Model market mid-price as Brownian motion with variance $\sigma^2$ (no drift)
    $$\mathrm d S_t=\sigma \mathrm d W_t,\;t\in[0,T]$$
    The dealer quotes bid $p^b$ and ask $p^a$ with spreads $\delta^b=s-p^b$ and $\delta^a=p^a-s$
    respectively.
    We assume that 
    \begin{itemize}
        \item Market buy orders will `lift' the dealer's sell orders at Poisson rate $\lambda^a(\delta^a)$ (a decreasing function of $\delta^a$)
        \item Market sell orders will `hit' the dealer's bid orders at rate $\lambda^b(\delta^b)$ (decreasing in $\delta^b$).
    \end{itemize}
    Have stochastic wealth and inventory:
    \begin{equation}
        \mathrm dX_t=p^a\mathrm dN_t^a-p^b\mathrm dN_t^b
    \end{equation}
    \begin{equation}
        q_t=N_t^b-N_t^a
    \end{equation}
\end{frame}

\begin{frame}{The Avellaneda-Stoikov model}
    Objective function: Maximise expected utility of terminal wealth over possible bid/ask spreads $\delta^a,\delta^b$
    \begin{equation}
        u(s,x,q,t)=\max_{\delta^a,\delta^b}\mathbb{E}\left[-e^{-\gamma(X_T+q_TS_T)}|\mathcal{F}_t\right]
    \end{equation}
    \begin{itemize}
        \item HJB equation very difficult (maybe impossible) to solve analytically
        \item Through some asymptotic approximations we can work out an approximate solution in terms of our model paramaters
    \end{itemize}
\end{frame}

\begin{frame}{The Avellaneda-Stoikov model}
    We obtain
    \begin{equation}
        r(s,q,t)=s-q\gamma\sigma^2(T-t)
    \end{equation}
    which coincides with our indifference price for the dealer with static inventory, and
    \begin{equation}
        \delta^a+\delta^b=\gamma\sigma^2(T-t)+\frac{2}{\gamma}\log\left(1+\frac{\gamma}{k}\right)
    \end{equation}
    with $\gamma$ and $\sigma$ as before, $k$ is a parameter from the orderbook describing how 
    market order size impacts prices.
\end{frame}

\begin{frame}{Statistical properties of the limit order book}
    Poisson intensity $\lambda(\delta)$ describes how likely a limit order is to be executed as a function of it's 
    distance to the mid-price. Need some statistics regarding:
    \begin{itemize}
        \item Overall frequency of market orders
        \begin{itemize}
            \item For simplicity, assume constant $\Lambda$
        \end{itemize}
        \item Size distribution of market orders
        \begin{itemize}
            \item ``Econophysics'' $\implies$ power law $f^Q(x)\propto x^{-1-\alpha}$
        \end{itemize}
        \item Price impact of large market orders
        \begin{itemize}
            \item ``Econophysics'' $\implies$ either $\Delta p\propto Q^\beta$ or $\Delta p\propto\log Q$
        \end{itemize}
    \end{itemize}
    Using the second result for price impact we obtain:
    \begin{align*}
        \lambda(\delta)=\Lambda\mathbb{P}(\Delta p>\delta)&=\Lambda\mathbb{P}(\log Q>K\delta)\\
        &=\Lambda\mathbb{P}\left(Q>e^{K\delta}\right)\\
        &=\Lambda\int_{e^{K\delta}}^{\infty}x^{-1-\alpha}dx\\
        &=Ae^{-k\alpha\delta}
    \end{align*}
\end{frame}

\begin{frame}{The Avellaneda-Stoikov model - Summary}
    \begin{itemize}
        \item Estimate parameters from market data
    \end{itemize}
    At each timestep, given current inventory and estimated parameters:
    \begin{itemize}
        \item Compute reservation price $r(s,q,t)$
        \item Compute spread $\delta^a+\delta^b$
        \item Set quotes $p^a=s+\frac{\delta^a+\delta^b}{2}$, $p^b=s-\frac{\delta^a+\delta^b}{2}$
    \end{itemize}
    \center{
        \includegraphics[height=5cm, width=7cm]{./plot.png}
    }
\end{frame}

% Conclusion
\begin{frame}{Conclusion}
    \begin{itemize}
        \item Hopefully you know a little bit more about how financial markets work!
        \item Stochastic control provides a useful framework through which to analyse the behaviour of market participants
        \item We can apply this framework to model the optimal behaviour of a dealer in financial markets
        \item This behaviour depends on the statistical properties of the particular market, which we can infer from market data
    \end{itemize}
\end{frame}

% Thank you & questions slide
\begin{frame}
    \frametitle{}
    \begin{center}
        \large{Thank you for your attention!}\\
        \vspace{1cm}
        Questions?
        \vspace{2cm}

        email: \texttt{josh.acton.2021@bristol.ac.uk}
    \end{center}
\end{frame}

\end{document}
