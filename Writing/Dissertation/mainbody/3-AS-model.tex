\section{Introduction}\label{sec:3.1}

In this chapter, we can finally return to the market-making problem introduced in
Chapter \ref{chap:1}, fully armed with the theory of stochastic optimal control 
that we have built up in Chapter \ref{chap:2}. We will formulate the problem and our
assumptions in the framework of \cite{AS2008}, and walk through
their methodology and theoretical results. 

We begin in section \ref{sec:3.2} by setting out our assumptions our assumptions about 
the dynamics of the market mid-point price and our agents utility function. In section 
\ref{sec:3.3}, we introduce the concept of an indifference or reservation price in the 
context of a passive agent with constant inventory, and derive some expressions. In 
\ref{sec:3.4}, we briefly analyse the infinite time horizon case, showing that 
analagous reservation prices exist, which may be of greater interest to dealers in 
markets that trade 24/7 such as FX and crypto. In section \ref{sec:3.5} we return to 
the finite horizon setting and define concepts such as market impact, arriving at the 
objective function of the agent who can set limit orders and thus influence the dynamics 
of their wealth over time. 

Of crucial importance to this
agent are the statistical properties of market orders: Their arrival frequency, the 
distribution of their size, and how they impact prices, which we discuss in section
\ref{sec:3.6}. Next we derive the Hamilton-Jacobi-Bellman equation in section \ref{sec:3.7},
and introduce an ansatz which allows us to simplify our problem and derive some useful
relations between the agents reservation price and optimal bid-ask spread. Finally,
we introduce some analytical approximations in section \ref{sec:3.8} that enable us to
derive an approximate solution in terms of our model parameters. 

The main result, which
we summarise in section \ref{sec:3.9}, is that optimal bid and ask quotes can be computed
through an intuitive two-step procedure: First, the agent computes a personal reservation
price for the asset, given her current inventory. Second, she calibrates her bid and ask
quotes to the limit order book, by considering the probability with which her quotes will
be executed as a function of their distance from the midpoint price.

\section{Model assumptions}\label{sec:3.2}

The paper of \cite{AS2008} is closely related to that of \cite{HS1981}, with the 
crucial difference being that while Ho and Stoll consider a monopolistic dealer,
Avellaneda and Stoikov consider a dealer who is potentially one of many dealers 
and many other market participants who may set limit orders. 

In Ho and Stoll (1981),
the authors specify a `true' price for the asset, and then allow the dealer to set 
quotes around this price. This may be more applicable to OTC markets in illiquid 
products where there is no openly accessible limit orderbook, but Avellaneda and Stoikov
consider a dealer operating in an openly accessible limit orderbook, and hence it makes
sense to view the mid-point price in the orderbook as the true price of the security. 

Another point to make here is that Avellaneda and Stoikov consider a dealer who is 
concerned only with inventory risk, not asymmetric information, and so assuming that
other market participants are better informed and reacting to fundamental arbitrage
opportunities, by the efficient market hypothesis we would have that the market mid
price is the best available measure of the true price of the asset given all the information
available up to a particular point in time.

We will assume that the market mid-point price evolves according to the SDE
\begin{equation}
    \mathrm dS_u=\sigma\mathrm dW_u
\end{equation}
with initial value $S_t=s$. $W_t$ is a standard one-dimensional Brownian motion, and
$\sigma>0$ is constant. Underlying this model is an implicit assumption that the agent
has no opinion on the drift or any autocorrelation or stochasticity of volatility 
for the stock. 

We also assume for simplicity that the money market pays no interest. Moreover, the
limit orders set by the agent can be continuously updated at no cost. In reality, 
the cost of trading will differ depending on the exchange in question, as most
charge a small percentage fee of every executed trade and some only charge market orders,
while providing rebates to dealers' trades for the liquidity they provide. Finally,
we assume that the lot sizes our limit orders are constant at one share per order,
and that the overall arrival frequency of market orders is constant.

We summarise our assumptions in the list below:

\begin{itemize}
    \item The dealer being modelled is one of many players in the market
    \item The `true' price is given by the market mid-price
    \item The mid-price evolves according to a brownian motion with constant volatility $\sigma$
    \item The agent has no opinion on drift or autocorrelation of the stock price
    \item The money-market pays no interest
    \item Limit orders can be continuously updated at no cost
    \item Limit orders are of fixed size 1
    \item The arrival frequency of market orders to the market is constant
\end{itemize}

\section{Modelling an inactive trader}\label{sec:3.3}

Our agents objective will be to maximise the expected utility of their wealth at 
a terminal time $T$. Avellaneda and Stoikov's choice of exponential utility is 
convenient since its convexity allows us to define reservation prices that are 
indepenent of the agents current wealth.

\subsection*{The utility function}

Initially, we consider an inactive trader who holds a fixed inventory of $q$ stocks 
until the terminal time $T.$ The agent's value function is
\begin{equation}\label{eq:3.1}
    v(x,s,q,t)=\mathbb{E}\left[-e^{-\gamma(x+qS_T)}|\mathcal{F}_t\right]
\end{equation}
where $x$ is the initial wealth in dollars, $t$ is the present time and $\gamma$ is 
a personal pre-defined risk-aversion parameter. By some simple manipulations, we can 
write this in a more convenient form as follows:
\begin{align*}
    v(x,s,q,t)&=\mathbb{E}\left[-e^{-\gamma(x+qS_T)}|\mathcal{F}_t\right]\\
    &=-e^{-\gamma x}\mathbb{E}\left[e^{-\gamma q S_T}|\mathcal{F}_t\right]\\
    &=-e^{-\gamma x}e^{-\gamma q s + \frac{\gamma^2q^2\sigma^2(T-t)}{2}}\\
    &=-e^{-\gamma x}e^{-\gamma q s}e^{\frac{\gamma^2q^2\sigma^2(T-t)}{2}}
\end{align*}

\subsection*{Reservation prices}

Following \cite{AS2008}, we can now use our value function to define the agents 
reservation bid and ask prices.
The reservation bid and reservation ask prices are simply the prices at which the agent
is indifferent between buying/selling and doing nothing. In other words, the reservation
bid (ask) is the price at which the agent is indifferent between her current portfolio
and her current portfolio $\pm$ one stock and $\mp$ the cash price.

\begin{definition}[Reservation bid price]
    Let $v$ be the value function of the agent. Its reservation bid price $r^b$ is
    given implicitly by the relation
    \begin{equation}\label{eq:3.2}
        v(x-r^b(s,q,t),s,q+1,t)=v(x,s,q,t)
    \end{equation}
    and the corresponding reservation ask price $r^a$ is similarly implicit in the 
    relation
    \begin{equation}\label{eq:3.3}
        v(x+r^a(s,q,t),s,q-1,t)=v(x,s,q,t).
    \end{equation}
\end{definition}

We can determine an exact expression for $r^b(s,q,t)$ by plugging our prior definition
for the value function, (\ref{eq:3.1}), in to our relation (\ref{eq:3.2}) as follows:
\begin{align*}
    v(x-r^b(s,q,t),s,q+1,t)&=v(x,s,q,t)\\
    -e^{-\gamma(x-r^b(s,q,t))}e^{-\gamma s(q+1)}e^{\frac{\gamma^2(q+1)^2\sigma^2(T-t)}{2}}&=-e^{-\gamma x}e^{-\gamma q s}e^{\frac{\gamma^2q^2\sigma^2(T-t)}{2}}\\
    -\gamma(x-r^b(s,q,t))-\gamma s(q+1) + \frac{\gamma^2(q+1)^2\sigma^2(T-t)}{2} &= -\gamma x-\gamma q s + \frac{\gamma^2q^2\sigma^2(T-t)}{2}\\
    \gamma r^b(s,q,t)-\gamma s + \frac{\gamma^2(1+2q)\sigma^2(T-t)}{2} &=0,
\end{align*}

dividing by $\gamma$ and rearranging to obtain
\begin{equation}
    r^b(s,q,t)=s+(-1-2q)\frac{\gamma\sigma^2(T-t)}{2}
\end{equation}
Similarly for $r^a(s,q,t)$:
\begin{align*}
    v(x+r^a(s,q,t),s,q-1,t)&=v(x,s,q,t)\\
    -e^{-\gamma(x+r^a(s,q,t))}e^{-\gamma s(q-1)}e^{\frac{\gamma^2(q-1)^2\sigma^2(T-t)}{2}}&=-e^{-\gamma x}e^{-\gamma q s}e^{\frac{\gamma^2q^2\sigma^2(T-t)}{2}}\\
    -\gamma(x+r^a(s,q,t))-\gamma s(q-1)+\frac{\gamma^2(q-1)^2\sigma^2(T-t)}{2}&=-\gamma x-\gamma q s + \frac{\gamma^2q^2\sigma^2(T-t)}{2}\\
    -\gamma r^a(s,q,t) + \gamma s + \frac{\gamma^2(1-2q)\sigma^2(T-t)}{2}&=0,
\end{align*}
again dividing by $\gamma$ and rearranging to obtain
\begin{equation}
    r^a(s,q,t)=s+(1-2q)\frac{\gamma\sigma^2(T-t)}{2}
\end{equation}
We define the \emph{reservation} or \emph{indifference} price to be the average of 
these two \textit{given} that the agent currently holds q stocks:
\begin{align*}
    r(s,q,t)&=\frac{r^a(s,q,t)+r^b(s,q,t)}{2}\\
    &=\frac{s+(1-2q)\frac{\gamma\sigma^2(T-t)}{2}+s+(-1-2q)\frac{\gamma\sigma^2(T-t)}{2}}{2}\\
    &=\frac{2s-2q\gamma\sigma^2(T-t)}{2}\\
    &=s-q\gamma\sigma^2(T-t)
\end{align*}
This price is nothing more than an adjustment to the mid-price which accounts for the 
effect of the inventory held by the agent on the agents preference to buy or sell. It
is easy to see that if the agent is long stock ($q>0$), the reservation price will be 
lower than the mid-price, reflecting the agents willingness to sell at a discount in 
order to reduce its inventory. Conversely, if the agent is short stock ($q<0$), its 
reservation price will be greater than the mid-price, indicating the agents preference
to buy at a premium to the market in order to return to a market-neutral position.

We note that the expressions derived above for $r^a$ and $r^b$ (and consequently $r$)
exist in the setting where $q$ is a fixed constant, and therefore it is not so simple
to derive these expressions when our agent is permitted to set limit orders. However,
they are important both as an illustrative example and because when we introduce our 
approximate solution in \ref{sec:3.8}, we will arrive at a very similar reservation price.

\section{The Optimising Agent with Infinite Horizon}\label{sec:3.4}
We will now briefly analyse the infinite horizon variant of the dealer problem, showing
that we can derive a stationary version of the reservation price through defining
an infinite horizon variant of our value function including a discount factor. This is
necessary since in our finite horizon case discussed above, our reservation price is
dependent upon the time interval $T-t.$ The intuition for this is that at or close to 
$T$, the agent may liquidate any remaining inventory for (or at least close to) $S_T$,
hence the closer time is to $T$, the less risk there is in the dealer's position.

We consider an infinite-horizon value function of the form
\begin{equation*}
    \bar v(x,s,q)=\mathbb{E}\left[\int_{0}^{\infty}-e^{-\omega t}e^{-\gamma(x+qS_t)}\mathrm dt\right]
\end{equation*}
where $\omega$ is our discount factor. An interpretation of $\omega$ is that it may
represent an upper bound on the absolute inventory position that the agent is allowed
to build up. A natural choice is to take $\omega=\frac{1}{2}\gamma^2\sigma^2(q_{\textrm{max}}+1)^2$
, this will be justified shortly. 

Using the definition of reservation bid and ask 
prices given above in section \ref{sec:3.3}, we can attain stationary versions of the 
reservation prices $r^b$ and $r^a$ with much the same method as before, only relying on
slightly more advanced theory, appealing to Tonelli's theorem (\ref{eq:1.11}) which 
allows us to swap the expectation and integral in the value function. For $r^b$, we 
have the following:
\begin{align*}
    \bar{v}(x-\bar{r}^b(s,q),s,q+1)&=\bar{v}(x,s,q)\\
    \mathbb{E}\left[\int_{0}^{\infty}-e^{-\omega t}e^{-\gamma(x-\bar{r}^b(s,q)+(q+1)S_t)}\mathrm dt\right]&=\mathbb{E}\left[\int_{0}^{\infty}-e^{-\omega t}e^{-\gamma(x+qS_t)}\mathrm dt\right]\\
    \int_{0}^{\infty}e^{-\omega t}e^{-\gamma(x-\bar{r}^b(s,q))}\mathbb{E}\left[e^{-\gamma(q+1)S_t}\right]\mathrm dt&=\int_{0}^{\infty}e^{-\omega t}e^{-\gamma x}\mathbb{E}\left[e^{-\gamma qS_t}\right]\mathrm dt\textrm{ (by Tonelli)}\\
    e^{-\gamma(x-\bar{r}^b(s,q))}\int_{0}^{\infty}e^{-\omega t}e^{-\gamma(q+1)s+\frac{\gamma^2(q-1)^2\sigma^2t}{2}}\mathrm dt&=e^{-\gamma x}\int_{0}^{\infty}e^{-\omega t}e^{-\gamma qs+\frac{\gamma^2q^2\sigma^2t}{2}}\mathrm dt\\
    e^{-\gamma(x-\bar{r}^b(s,q))}e^{-\gamma(q+1)s}\int_{0}^{\infty}e^{-\omega t}e^{\frac{\gamma^2(q+1)^2\sigma^2t}{2}}\mathrm dt&=e^{-\gamma x}e^{-\gamma qs}\int_{0}^{\infty}e^{-\omega t}e^{\frac{\gamma^2q^2\sigma^2t}{2}}\mathrm dt\\
    e^{\gamma \bar{r}^b(s,q)}e^{-\gamma s}\int_{0}^{\infty}e^{\left(\frac{\gamma^2(q+1)^2\sigma^2-2\omega}{2}\right)t}\mathrm dt&=\int_{0}^{\infty}e^{\left(\frac{\gamma^2q^2\sigma^2-2\omega}{2}\right)t}\mathrm dt\\
    e^{\gamma \bar{r}^b(s,q)}e^{-\gamma s}\left(\frac{2}{2\omega-\gamma^2(q+1)^2\sigma^2}\right)&=\left(\frac{2}{2\omega-\gamma^2q^2\sigma^2}\right)\\
    e^{\gamma(\bar{r}^b(s,q)-s)}&=\frac{2\omega-\gamma^2(q+1)^2\sigma^2}{2\omega-\gamma^2q^2\sigma^2}\\
    e^{\gamma(\bar{r}^b(s,q)-s)}&=1-\frac{(1+2q)\gamma^2\sigma^2}{2\omega-\gamma^2q^2\sigma^2}\\
    \gamma\bar{r}^b(s,q)-\gamma s&=\log\left(1+\frac{(-1-2q)\gamma^2\sigma^2}{2\omega-\gamma^2q^2\sigma^2}\right)\\
    \bar{r}^b(s,q) &= s+\frac{1}{\gamma}\log\left(1+\frac{(-1-2q)\gamma^2\sigma^2}{2\omega-\gamma^2q^2\sigma^2}\right)
\end{align*}
which is valid for $\omega>\frac{1}{2}\gamma^2\sigma^2q^2$ and agrees exactly with
the result presented in \cite{AS2008}. We can now perform the same procedure for the
reservation ask price $r^a$:
\begin{align*}
    \bar{v}(x+r^a(s,q),s,q-1)&=\bar{v}(x,s,q)\\
    \mathbb{E}\left[\int_{0}^{\infty}-e^{-\omega t}e^{-\gamma(x+r^a(s,q)+(q-1)S_t)}\mathrm dt\right]&=\mathbb{E}\left[\int_{0}^{\infty}-e^{-\omega t}e^{-\gamma(x+qS_t)}\mathrm dt\right]\\
    \int_{0}^{\infty}e^{-\omega t}e^{-\gamma(x+r^a(s,q))}\mathbb{E}\left[e^{-\gamma(q-1)S_t}\right]\mathrm dt&=\int_{0}^{\infty}e^{-\omega t}e^{-\gamma x}\mathbb{E}\left[e^{-\gamma qS_t}\right]\mathrm dt\textrm{ (by Tonelli)}\\
    e^{-\gamma(x+r^a(s,q))}\int_{0}^{\infty}e^{-\omega t}e^{-\gamma(q-1)s+\frac{\gamma^2(q-1)^2\sigma^2t}{2}}\mathrm dt&=e^{-\gamma x}\int_{0}^{\infty}e^{-\omega t}e^{-\gamma qs+\frac{\gamma^2q^2\sigma^2t}{2}}\mathrm dt\\
    e^{-\gamma(x+r^a(s,q))}e^{-\gamma(q-1)s}\int_{0}^{\infty}e^{-\omega t}e^{\frac{\gamma^2(q-1)^2\sigma^2t}{2}}\mathrm dt&=e^{-\gamma x}e^{-\gamma qs}\int_{0}^{\infty}e^{-\omega t}e^{\frac{\gamma^2q^2\sigma^2t}{2}}\mathrm dt\\
    e^{-\gamma r^a(s,q)}e^{\gamma s}\int_{0}^{\infty}e^{\left(\frac{\gamma^2(q-1)^2\sigma^2-2\omega}{2}\right)t}\mathrm dt&=\int_{0}^{\infty}e^{\left(\frac{\gamma^2q^2\sigma^2-2\omega}{2}\right)t}\mathrm dt\\
    e^{-\gamma r^a(s,q)}e^{\gamma s}\left(\frac{2}{2\omega-\gamma^2(q-1)^2\sigma^2}\right)&=\left(\frac{2}{2\omega-\gamma^2q^2\sigma^2}\right)\\
    e^{\gamma(s-r^a(s,q))}&=\frac{2\omega-\gamma^2(q-1)^2\sigma^2}{2\omega-\gamma^2q^2\sigma^2}\\
    e^{\gamma(s-r^a(s,q))}&=1-\frac{(1-2q)\gamma^2\sigma^2}{2\omega-\gamma^2q^2\sigma^2}\\
    \gamma s - \gamma r^a(s,q)&=\log\left(1-\frac{(1-2q)\gamma^2\sigma^2}{2\omega-\gamma^2q^2\sigma^2}\right)\\
    r^a(s,q) &= s-\frac{1}{\gamma}\log\left(1-\frac{(1-2q)\gamma^2\sigma^2}{2\omega-\gamma^2q^2\sigma^2}\right)
\end{align*}
which is again valid for $\omega>\frac{1}{2}\gamma^2\sigma^2q^2$. However, this differs
from the result presented in \cite{AS2008}, which they give to be:
\begin{equation*}
    r^a(s,q)=s+\frac{1}{\gamma}\log\left(1+\frac{(1-2q)\gamma^2\sigma^2}{2\omega-\gamma^2q^2\sigma^2}\right)
\end{equation*}
Using the correct values of the reservation prices obtained above, we can derive an
expression for the reservation price as we did before.
\begin{align*}
    r(s,q):=&\frac{r^a(s,q)+r^b(s,q)}{2}\\
    =&\frac{s-\frac{1}{\gamma}\log\left(1-\frac{(1-2q)\gamma^2\sigma^2}{2\omega-\gamma^2q^2\sigma^2}\right)+s+\frac{1}{\gamma}\log\left(1+\frac{(-1-2q)\gamma^2\sigma^2}{2\omega-\gamma^2q^2\sigma^2}\right)}{2}\\
    =&s+\frac{1}{2\gamma}\left(\log\left(1+\frac{(-1-2q)\gamma^2\sigma^2}{2\omega-\gamma^2q^2\sigma^2}\right)-\log\left(1-\frac{(1-2q)\gamma^2\sigma^2}{2\omega-\gamma^2q^2\sigma^2}\right)\right)\\
    =&s+\frac{1}{2\gamma}\log\left(\frac{1-\frac{(1+2q)\gamma^2\sigma^2}{2\omega-\gamma^2q^2\sigma^2}}{1-\frac{(1-2q)\gamma^2\sigma^2}{2\omega-\gamma^2q^2\sigma^2}}\right)\\
    =&s+\frac{1}{2\gamma}\log\left(\frac{1-\frac{(1-2q)\gamma^2\sigma^2+4q\gamma^2\sigma^2}{2\omega-\gamma^2q^2\sigma^2}}{1-\frac{(1-2q)\gamma^2\sigma^2}{2\omega-\gamma^2q^2\sigma^2}}\right)\\
    =&s+\frac{1}{2\gamma}\log\left(1+\frac{\frac{4q\gamma^2\sigma^2}{2\omega-\gamma^2q^2\sigma^2}}{1-\frac{(1-2q)\gamma^2\sigma^2}{2\omega-\gamma^2q^2\sigma^2}}\right)\\
    =&s+\frac{1}{2\gamma}\log\left(1+\frac{4q\gamma^2\sigma^2}{2\omega-\gamma^2q^2\sigma^2-(1-2q)\gamma^2\sigma^2}\right)
\end{align*}

\section{Modelling Limit Orders}\label{sec:3.5}
The agent quotes the bid price $p^b$ and the ask price $p^a$, and the current shape of the limit orderbook as well as the distances 
$$\delta^b:=s-p^b$$
and 
$$\delta^a:=p^a-s$$ 
determine the priority of execution when large market orders are placed. 
E.g. when a market order to buy $Q$ shares arrives, the $Q$ limit orders with the lowest ask prices will be lifted. 
Let $p^Q$ be the price of the highest limit order executed in this trade. Then 
$$\Delta p:=p^Q-s$$ 
is the temporary market impact of the trade of size $Q.$ 
Then we have that if our $\delta^a < \Delta p$, our agents limit order will be executed. We assume that market orders will fill our limit orders at Poisson rates $\lambda^a(\delta^a)$ and $\lambda^b(\delta^b)$, decreasing functions of $\delta^a$ and $\delta^b$ resp. 
(further away from midpoint $\rightarrow$ orders hit less often).

We now have stochastic wealth and inventory: Let $N^b_t$ and $N^a_t$ be Poisson processes with intensities $\lambda^b$ and $\lambda^a$ representing the amount of stocks bought/sold by the agent at time t. The inventory at time t is 
$$q_t=N^b_t-N^a_t$$ and the wealth process evolves according to
$$dXt=p^adN^a_t-p^bdN^b_t.$$
The objective of the agent who sets limit orders is 
$$u(s,x,q,t)=\max\limits_{\delta^a,\delta^b}\mathbb{E}_t\left[-e^{-\gamma(X_T+q_TS_T)}\right]$$

\section{Modelling Trading Intensity}\label{sec:3.6}
Assume constant frequency $\Lambda$ of market orders. We want to determine some realistic functional forms for the relationship between the Poisson intensity $\lambda$ and distance to mid-price $\delta$. 
To do this we need information on: (i) the overall frequency of market orders, (ii) the distribution of their size, (iii) the temporary impact of a large market order. 
\subsection*{Distribution of the size of market orders}
The distribution of size of market orders has been found to obey a power law:
\begin{equation}\label{eq:2}
    f^{Q}(x)\propto x^{-1-\alpha}
\end{equation}
for large $x$, with $\alpha\in[1.4,1.6].$
\subsection*{Modelling market impact}
Less consensus on market impact. Some find change in price $\Delta p$ after market order size $Q$ given by 
\begin{equation}\label{eq:3}
    \Delta p\propto Q^\beta, \beta\in[0.5,0.8]
\end{equation}
while others find
\begin{equation}\label{eq:4}
    \Delta p\propto\log(Q)
\end{equation}
Using \ref{eq:2} and \ref{eq:4} we can derive the poisson intensity as follows:
\begin{align*}
    \lambda(\delta)&=\Lambda\mathbb{P}(\delta<\Delta p)\\
    &=\Lambda\mathbb{P}\left(\delta<\frac{\log Q}{K}\right)\\
    &=\Lambda\mathbb{P}(K\delta<\log Q)\\
    &=\Lambda\mathbb{P}\left(e^{K\delta}<Q\right)
    &=\Lambda\int_{e^{K\delta}}^{\infty}x^{-1-\alpha}dx\\
    &=\Lambda\left[\frac{-x^{-\alpha}}{\alpha}\right]_{e^{K\delta}}^\infty\\
    &=\Lambda\left(\lim_{t\rightarrow\infty}\frac{-t^{-\alpha}}{\alpha}+\frac{e^{-K\delta\alpha}}{\alpha}\right)\\
    &=\frac{\Lambda}{\alpha}\left(e^{-K\delta\alpha}-\lim_{t\rightarrow\infty}\frac{1}{t^\alpha}\right)\\
    &=\frac{\Lambda}{\alpha}e^{-\alpha K\delta}
\end{align*}
while \ref{eq:2} and \ref{eq:3} yield:
\begin{align*}
    \lambda(\delta)&=\Lambda\mathbb{P}(\delta<\Delta p)\\
    &=\Lambda\mathbb{P}(\delta<kQ^\beta)\\
    &=\Lambda\mathbb{P}\left(Q>\left(\frac{\delta}{k}\right)^{-\beta}\right)\\
    &=\Lambda\int_{\left(\frac{\delta}{k}\right)^{-\beta}}^\infty x^{-1-\alpha}dx\\
    &=\Lambda\left[\lim_{t\rightarrow\infty}\frac{-t^{-\alpha}}{\alpha}+\frac{\left(\frac{\delta}{k}\right)^{-\frac{\alpha}{\beta}}}{\alpha}\right]\\
    &=\frac{\Lambda\left(\frac{\delta}{k}\right)^{-\frac{\alpha}{\beta}}}{\alpha}
\end{align*}
Other methods exist i.e. integrating the density of the orderbook, potentially better since we only care abt short-term liquidity?

\section{The Hamilton-Jacobi-Bellman Equation}\label{sec:3.7}
Ho and stoll use the dynamic programming principle to show that a function $u$ must solve the HJB:
\begin{equation}\label{eq:hjb-1}
    \left\{
        \begin{aligned}
            u_t+\frac{1}{2}\sigma^2u_{ss}&+\max\limits_{\delta^b}\lambda^b(\delta^b)[u(s,x-s+\delta^b,q+1,t)-u(s,x,q,t)]\\
            &+\max\limits_{\delta^a}\lambda^a(\delta^a)[u(s,x+s+\delta^a,q-1,t)-u(s,x,q,t)]=0,\\
            &u(s,x,q,T)=-e^{-\gamma(x+qs)}
        \end{aligned}
    \right.
\end{equation}
but due to our choice of exponential utility we can simplify the problem with the ansatz:
$$u(s,x,q,t)=-e^{-\gamma x}e^{-\gamma\theta(s,q,t)}$$
and by substitution we find the following equation for $\theta$:
\begin{equation}\label{eq:hjb-2}
    \left\{
        \begin{aligned}
            \theta_t+\frac{1}{2}\sigma^2\theta_{ss}-\frac{1}{2}\sigma^2\gamma\theta_{ss}^2&+\max\limits_{\delta^b}\left[\frac{\lambda^b(\delta^b)}{\gamma}(1-e^{\gamma(s-\delta^b-r^b)})\right]\\
            &+\max\limits_{\delta^a}\left[\frac{\lambda^a(\delta^a)}{\gamma}(1-e^{-\gamma(s+\delta^a-r^a)})\right]=0,\\
            &\theta(s,q,T)=qs.
        \end{aligned}
    \right.
\end{equation}
\subsection*{Relations for the reserve prices}
By the definitions of the reserve bid and ask prices we obtain
\begin{align*}
    u(s,x-r^b(s,q,t),q+1,t)&=u(s,x,q,t)\\
    -e^{-\gamma(x-r^b(s,q,t))}e^{-\gamma\theta(s,q+1,t)}&=-e^{-\gamma x}e^{-\gamma\theta(s,q,t)}\\
    -\gamma(x-r^b(s,q,t))-\gamma\theta(s,q+1,t)&=-\gamma x-\gamma\theta(s,q,t)\\
    x-r^b(s,q,t)+\theta(s,q+1,t)&=x+\theta(s,q,t)\\
    r^b(s,q,t)&=\theta(s,q+1,t)-\theta(s,q,t)
\end{align*}
and
\begin{align*}
    u(s,x+r^a(s,q,t),q-1,t)&=u(s,x,q,t)\\
    -e^{-\gamma(x+r^a(s,q,t))}e^{-\gamma\theta(s,q-1,t)}&=-e^{-\gamma x}e^{-\gamma\theta(s,q,t)}\\
    -\gamma(x+r^a(s,q,t))-\gamma\theta(s,q-1,t)&=-\gamma x-\gamma\theta(s,q,t)\\
    x+r^a(s,q,t)+\theta(s,q-1,t)&=x+\gamma\theta(s,q,t)\\
    r^a(s,q,t)&=\theta(s,q,t)-\theta(s,q-1,t)
\end{align*}
From the first order optimality condition on the maximised terms in the HJB equation, we may obtain as follows some implicit relations
on the optimal bid and ask spreads $\delta^b$ and $\delta^a$:
\begin{align*}
    \frac{\partial}{\partial\delta}\left[\frac{\lambda^b(\delta)}{\gamma}(1-e^{\gamma(s-\delta-r^b(s,q,t))})\right](\delta^b)&=0\\
    \frac{1}{\gamma}\left[\frac{\partial\lambda^b}{\partial\delta}(\delta^b)-\frac{\partial}{\partial\delta}\lambda^b(\delta^b)e^{\gamma(s-\delta^b-r^b(s,q,t))}\right]&=0\\
    \frac{\partial\lambda^b}{\partial\delta}(\delta^b)-\frac{\partial\lambda^b}{\partial\delta}(\delta^b)e^{\gamma(s-\delta^b-r^b(s,q,t))}+\gamma\lambda^b(\delta^b)e^{\gamma(s-\delta^b-r^b(s,q,t))}&=0\\
    \left(\gamma\lambda^b(\delta^b)-\frac{\partial\lambda^b}{\partial\delta}(\delta^b)\right)e^{\gamma(s-\delta^b-r^b(s,q,t))}&=-\frac{\partial\lambda^b}{\partial\delta}(\delta^b)\\
    -\left(\frac{\partial\lambda^b}{\partial\delta}(\delta^b)\right)e^{-\gamma(s-\delta^b-r^b(s,q,t))}&=\gamma\lambda^b(\delta^b)-\frac{\partial\lambda^b}{\partial\delta}(\delta^b)\\
    e^{-\gamma(s-\delta^b-r^b(s,q,t))}&=1-\gamma\frac{\lambda^b(\delta^b)}{\frac{\partial\lambda^b}{\partial\delta}(\delta^b)}\\
    -\gamma(s-\delta^b-r^b(s,q,t))&=\log\left(1-\gamma\frac{\lambda^b(\delta^b)}{\frac{\partial\lambda^b}{\partial\delta}(\delta^b)}\right)\\
    s-\delta^b-r^b(s,q,t)&=-\frac{1}{\gamma}\log\left(1-\gamma\frac{\lambda^b(\delta^b)}{\frac{\partial\lambda^b}{\partial\delta}(\delta^b)}\right)\\
    s-r^b(s,q,t)&=\delta^b-\frac{1}{\gamma}\log\left(1-\gamma\frac{\lambda^b(\delta^b)}{\frac{\partial\lambda^b}{\partial\delta}(\delta^b)}\right)
\end{align*}
and
\begin{align*}
    \frac{\partial}{\partial\delta}\left[\frac{\lambda^a(\delta)}{\gamma}(1-e^{-\gamma(s+\delta-r^a(s,q,t))})\right](\delta^a)&=0\\
    \frac{1}{\gamma}\left[\frac{\partial\lambda^a}{\partial\delta}(\delta^a)-\frac{\partial}{\partial\delta}\lambda^a(\delta^a)e^{-\gamma(s+\delta^a-r^a(s,q,t))}\right]&=0\\
    \frac{\partial\lambda^a}{\partial\delta}(\delta^a)-\frac{\partial\lambda^a}{\partial\delta}(\delta^a)e^{-\gamma(s+\delta^a-r^a(s,q,t))}+\gamma\lambda^a(\delta^a)e^{-\gamma(s+\delta^a-r^a(s,q,t))}&=0\\
    \left(\gamma\lambda^a(\delta^a)-\frac{\partial\lambda^a}{\partial\delta}(\delta^a)\right)e^{-\gamma(s+\delta^a-r^a(s,q,t))}&=-\frac{\partial\lambda^a}{\partial\delta}(\delta^a)\\
    -\left(\frac{\partial\lambda^a}{\partial\delta}(\delta^a)\right)e^{\gamma(s+\delta^a-r^a(s,q,t))}&=\gamma\lambda^a(\delta^a)-\frac{\partial\lambda^a}{\partial\delta}(\delta^a)\\
    e^{\gamma(s+\delta^a-r^a(s,q,t))}&=1-\gamma\frac{\lambda^a(\delta^a)}{\frac{\partial\lambda^a}{\partial\delta}(\delta^a)}\\
    \gamma(s+\delta^a-r^a(s,q,t))&=\log\left(1-\gamma\frac{\lambda^a(\delta^a)}{\frac{\partial\lambda^a}{\partial\delta}(\delta^a)}\right)\\
    s+\delta^a-r^a(s,q,t)&=\frac{1}{\gamma}\log\left(1-\gamma\frac{\lambda^a(\delta^a)}{\frac{\partial\lambda^a}{\partial\delta}(\delta^a)}\right)\\
    r^a(s,q,t)-s&=\delta^a-\frac{1}{\gamma}\log\left(1-\gamma\frac{\lambda^a(\delta^a)}{\frac{\partial\lambda^a}{\partial\delta}(\delta^a)}\right)
\end{align*}

\section{Asymptotic Expansion in q}\label{sec:3.8}
\begin{equation}
    \lambda^a(\delta)+\lambda^b(\delta)=Ae^{-k\delta}
\end{equation}

Using exponential arrival rates in our relations above for the optimal bid and 
ask prices, we can see that
\begin{equation*}
    \frac{\lambda(\delta)}{\frac{\partial\lambda}{\partial\delta}(\delta)}=\frac{Ae^{-k\delta}}{-kAe^{-k\delta}}=-\frac{1}{k}.
\end{equation*}
Hence by plugging in the aforementioned relations under the assumption of symmetric
exponential arrival rates to the maximised terms in the HJB equation \ref{eq:hjb-2}
we see that
\begin{align*}
    &\max\limits_{\delta^b}\left[\frac{\lambda^b(\delta^b)}{\gamma}(1-e^{\gamma(s-\delta^b-r^b)})\right]+\max\limits_{\delta^a}\left[\frac{\lambda^a(\delta^a)}{\gamma}(1-e^{-\gamma(s+\delta^a-r^a)})\right]\\
    &=\frac{Ae^{-k\delta^b}}{\gamma}\left(1-e^{\gamma\left(-\frac{1}{\gamma}\log\left(1+\frac{\gamma}{k}\right)\right)}\right)+\frac{Ae^{-k\delta^a}}{\gamma}\left(1-e^{-\gamma\left(\frac{1}{\gamma}\log\left(1+\frac{\gamma}{k}\right)\right)}\right)\\
    &=\left[\frac{A}{\gamma}\left(1-e^{-\log\left(1+\frac{\gamma}{k}\right)}\right)\right](e^{-k\delta^b}+e^{-k\delta^a})\\
    &=\left[\frac{A}{\gamma}\left(1-\frac{1}{1+\frac{\gamma}{k}}\right)\right](e^{-k\delta^b}+e^{-k\delta^a})\\
    &=\left(\frac{A}{\gamma}-\frac{A}{\gamma+\frac{\gamma^2}{k}}\right)(e^{-k\delta^b}+e^{-k\delta^a})\\
    &=\left(\frac{A\left(1+\frac{\gamma}{k}\right)-A}{\gamma+\frac{\gamma^2}{k}}\right)(e^{-k\delta^b}+e^{-k\delta^a})\\
    &=\left(\frac{A\frac{\gamma}{k}}{\gamma+\frac{\gamma^2}{k}}\right)(e^{-k\delta^b}+e^{-k\delta^a})\\
    &=\frac{A}{k+\gamma}(e^{-k\delta^b}+e^{-k\delta^a})
\end{align*}
which results in the simplified HJB equation below:
\begin{equation}\label{eq:simp-hjb}
    \begin{cases} 
        \theta_t+\frac{1}{2}\sigma^2\theta_{ss}-\frac{1}{2}\sigma^2\gamma\theta^2_s+\frac{A}{k+\gamma}(e^{-k\delta^a}+e^{-k\delta^b})=0\\
        \theta(s,q,T)=qs.
    \end{cases}
\end{equation}
Asymptotic expansion in the inventory variable $q$:
\begin{equation}\label{eq:asymp}
    \theta(q,s,t)=\theta^0(s,t)+q\theta^1(s,t)+\frac{1}{2}q^2\theta^2(s,t)+...
\end{equation}
The exact relations for the reserve bid and ask prices obtained above yield
\begin{gather}
    r^b(s,q,t)=\theta^1(s,t)+(1+2q)\theta^2(s,t)+...\\
    r^a(s,q,t)=\theta^1(s,t)+(-1-2q)\theta^2(s,t)+...\;.
\end{gather}
Then
\begin{equation}\label{eq:reserve-price}
    r(s,q,t)=\frac{r^a(s,q,t)+r^b(s,q,t)}{2}=\theta^1(s,t)+2q\theta^2(s,t)
\end{equation}
follows immediately, and we also have that
\begin{equation}\label{eq:spread}
    \begin{aligned}
        \delta^a+\delta^b&=\frac{1}{\gamma}\log\left(1+\frac{\gamma}{k}\right)+r^a(s,q,t)-s+\frac{1}{\gamma}\log\left(1+\frac{\gamma}{k}\right)+s-r^b(s,q,t)\\
        &=r^a(s,q,t)-r^b(s,q,t)+\frac{2}{\gamma}\log\left(1+\frac{\gamma}{k}\right)\\
        &=-2\theta^2(s,t)+\frac{2}{\gamma}\log\left(1+\frac{\gamma}{k}\right)
    \end{aligned}
\end{equation}
Now consider a first-order approximation of the order arrival term:
\begin{equation}\label{eq:approx-arrival}
    \frac{A}{k+\gamma}(e^{-\gamma\delta^a}+e^{-\gamma\delta^b})=\frac{A}{k+\gamma}(2-k(\delta^a+\delta^b)+...)
\end{equation}
The linear term does not depend on the inventory $q$. Therefore, by substituting
\ref{eq:asymp} and \ref{eq:approx-arrival} into \ref{eq:simp-hjb} and 
grouping terms of order $q$ we obtain
\begin{equation}
    \begin{cases} 
        \theta^1_t+\frac{1}{2}\sigma^2\theta^1_{ss}=0\\
        \theta^1(s,T)=s.
    \end{cases}
\end{equation}
which admits the solution $\theta^1(s,t)=s$. Grouping terms of order $q^2$ yields
\begin{equation}
    \begin{cases}
        \theta^2_t+\frac{1}{2}\sigma^2\theta^2_{ss}-\frac{1}{2}\sigma^2\gamma(\theta^1_s)^2=0\\
        \theta^2(s,T)=0
    \end{cases}
\end{equation}
with solution $\theta^2(s,t)=-\frac{1}{2}\sigma^2\gamma(T-t)$.
Thus for this linear approximation of the order arrival term, we can substitute our
solutions back into \ref{eq:reserve-price} to obtain the same indifference price
\begin{equation}
    r(s,t)=s-q\gamma\sigma^2(T-t)
\end{equation}
as in the case where no trading is allowed. We quote a bid-ask spread given by
\begin{equation}
    \delta^a+\delta^b=\gamma\sigma^2(T-t)+\frac{2}{\gamma}\log\left(1+\frac{\gamma}{k}\right)
\end{equation}
which is again aquired by substituting our solutions back into \ref{eq:spread}.

\section{Summary}\label{sec:3.9}
The reserve ask price derived by Avellaneda and Stoikov in the infinite horizon case
is potentially incorrect, a correction is presented.