In this section we will walk through the methodology and theoretical
results of Avellaneda and Stoikov \cite{as}

\begin{itemize}
    \item The dealer being modelled is one of many players in the market
    \item The `true' price is given by the market mid-price
    \item The money-market pays no interest
    \item The agent has no opinion on drift or autocorrelation of the stock price
    \item Limit orders can be continuously updated at no cost
    \item The arrival frequency of market orders to the market is constant
    \item Limit orders are of fixed size 1
\end{itemize}

Assume the stock evolves according to a standard Wiener process with some variance $\sigma^2$: 
$$dS_u=\sigma dW_u$$

Consider an inactive trader who holds an inventory of $q$ stocks until the terminal time $T.$ The agent's value function is
$$v(x,s,q,t)=\mathbb{E}_t\left(-e^{-\gamma(x+qS_T)}\right)$$
where $x$ is the initial wealth in dollars, $t$ is the present time and $\gamma$ is a user-defined risk-aversion parameter.

The reservation price is the price that would make the agent indifferent between his current portfolio and his current portfolio plus one stock. So $r^b$ can be determined from the relation
$$v(x-r^b(s,q,t),s,q+1,t)=v(x,s,q,t)$$
and $r^a$ solves
$$v(x+r^a(s,q,t),s,q-1,t)=v(x,s,q,t).$$
We solve these to obtain
$$r^a(s,q,t)=s+(1-2q)\frac{\gamma\sigma^2(T-t)}{2}$$
and
$$r^b(s,q,t)=s+(-1-2q)\frac{\gamma\sigma^2(T-t)}{2}.$$
We define the ``reserve" or ``indifference" price to be the average of these two \textit{given} that the agent currently holds q stocks:
$$r(s,q,t)=s-q\gamma\sigma^2(T-t)$$

The agent quotes the bid price $p^b$ and the ask price $p^a$, and the current shape of the limit orderbook as well as the distances $\delta^b=s-p^b$ and $\delta^a=p^a-s$ determine the priority of execution when large market orders are placed. 
E.g. when a market order to buy $Q$ shares arrives, the $Q$ limit orders with the lowest ask prices will be lifted. 
Let $p^Q$ be the price of the highest limit order executed in this trade. Then $\Delta p=p^Q-s$ is the temporary market impact of the trade of size $Q.$ 
Then we have that if our $\delta^a < \Delta p$, our agents limit order will be executed. We assume that market orders will fill our limit orders at Poisson rates $\lambda^a(\delta^a)$ and $\lambda^b(\delta^b)$, decreasing functions of $\delta^a$ and $\delta^b$ resp. 
(further away from midpoint $\rightarrow$ orders hit less often).

We now have stochastic wealth and inventory: Let $N^b_t$ and $N^a_t$ be Poisson processes with intensities $\lambda^b$ and $\lambda^a$ representing the amount of stocks bought/sold by the agent at time t. The inventory at time t is $q_t=N^b_t-N^a_t$ and the wealth process evolves according to
$$dXt=p^adN^a_t-p^bdN^b_t.$$
The objective of the agent who sets limit orders is 
$$u(s,x,q,t)=\max\limits_{\delta^a,\delta^b}\mathbb{E}_t\left[-e^{-\gamma(X_T+q_TS_T)}\right]$$

Assume constant frequency $\Lambda$ of market orders. We want to determine some realistic functional forms for the relationship between the Poisson intensity $\lambda$ and distance to mid-price $\delta$. 
To do this we need information on: (i) the overall frequency of market orders, (ii) the distribution of their size, (iii) the temporary impact of a large market order. The distribution of size of market orders has been found to obey a power law:
\begin{equation}
    f^{Q}(x)\propto x^{-1-\alpha}
\end{equation}
for large $x$, with $\alpha\in[1.4,1.6].$
Less consensus on size distribution. Some find change in price $\Delta p$ after market order size $Q$ given by 
\begin{equation}
    \Delta p\propto Q^\beta, \beta\in[0.5,0.8]
\end{equation}
while others find
\begin{equation}
    \Delta p\propto\log(Q)
\end{equation}
Using (1) and (3) we can derive the poisson intensity as
$$\lambda(\delta)=\frac{\Lambda}{\alpha}e^{-\alpha K\delta}$$
while (1) and (2) yield
$$\lambda(\delta)=B\delta^{\frac{-\alpha}{\beta}}.$$
[Need to figure out what B is]. Other methods exist i.e. integrating the density of the orderbook, potentially better since we only care abt short-term liquidity?

Ho and stoll use the dynamic programming principle to show that a function $u$ must solve the HJB:
\begin{align*}
    u_t + \frac{1}{2} \sigma^2u_{ss} &+ \max \limits_{ \delta^b } \lambda^b( \delta^b )[ u(s,x-s+\delta^b,q+1,t)-u(s,x,q,t)] \\
    &+\max\limits_{\delta^a}\lambda^a(\delta^a)[u(s,x+s+\delta^a,q-1,t)-u(s,x,q,t)]=0, \\
    & u(s,x,q,T) = -e^{-\gamma(x+qs)}
\end{align*}
but due to our choice of exponential utility we can simplify the problem with the ansatz:
$$u(s,x,q,t)=-e^{-\gamma x}e^{-\gamma\theta(s,q,t)}$$
and by substitution we find the following equation for $\theta:$
\begin{align*}
    \theta_t+\frac{1}{2}\sigma^2\theta_{ss}-\frac{1}{2}\sigma^2\gamma\theta_{ss}^2&+\max\limits_{\delta^b}\left[\frac{\lambda^b(\delta^b)}{\gamma}(1-e^{\gamma(s-\delta^b-r^b)})\right]\\
    &+\max\limits_{\delta^a}\left[\frac{\lambda^a(\delta^a)}{\gamma}(1-e^{-\gamma(s+\delta^a+r^a)})\right]=0,\\
    &\theta(s,q,T)=qs.
\end{align*}
By the definitions of the reserve bid and ask prices we obtain
$$r^b(s,q,t)=\theta(s,q+1,t)-\theta(s,q,t)$$
and
$$r^a(s,q,t)=\theta(s,q,t)-\theta(s,q-1,t)$$
and then through the following implicit relation we can obtain the optimal distances $\delta^b$ and $\delta^a$:
$$s-r^b(s,q,t)=\delta^b-\frac{1}{\gamma}\log\left(1-\gamma\frac{\lambda^b(\delta^b)}{\frac{\partial\lambda^b}{\partial\delta}(\delta^b)}\right)$$
and
$$r^a(s,q,t)-s=\delta^a-\frac{1}{\gamma}\log\left(1-\gamma\frac{\lambda^a(\delta^a)}{\frac{\partial\lambda^a}{\partial\delta}(\delta^a)}\right).$$

\section{The Model}
\section{The Utility Function}
\section{The Optimising Agent with Finite Horizon}
\section{The Optimising Agent with Infinite Horizon}
\section{Modelling Limit Orders}
\section{Modelling Trading Intensity}
\section{The HJB}
\section{Asymptotic Expansion in q}
\section{Summary}