\section{Introduction}\label{sec:3.1}

In this chapter, we can finally return to the market-making problem introduced in
Chapter \ref{chap:1}, fully armed with the theory of stochastic optimal control 
that we have built up in Chapter \ref{chap:2}. We will formulate the problem and our
assumptions in the framework of \cite{AS2008}, and walk through
their methodology and theoretical results. 

We begin in section \ref{sec:3.2} by setting out our assumptions our assumptions about 
the dynamics of the market mid-point price and our agents utility function. In section 
\ref{sec:3.3}, we introduce the concept of an indifference or reservation price in the 
context of a passive agent with constant inventory, and derive some expressions. In 
\ref{sec:3.4}, we briefly analyse the infinite time horizon case, showing that 
analagous reservation prices exist, which may be of greater interest to dealers in 
markets that trade 24/7 such as FX and crypto. In section \ref{sec:3.5} we return to 
the finite horizon setting and define concepts such as market impact, arriving at the 
objective function of the agent who can set limit orders and thus influence the dynamics 
of their wealth over time. 

Of crucial importance to this
agent are the statistical properties of market orders: Their arrival frequency, the 
distribution of their size, and how they impact prices, which we discuss in section
\ref{sec:3.6}. Next we derive the Hamilton-Jacobi-Bellman equation in section \ref{sec:3.7},
and introduce an ansatz which allows us to simplify our problem and derive some useful
relations between the agents reservation price and optimal bid-ask spread. Finally,
we introduce some analytical approximations in section \ref{sec:3.8} that enable us to
derive an approximate solution in terms of our model parameters. 

The main result, which
we summarise in section \ref{sec:3.9}, is that optimal bid and ask quotes can be computed
through an intuitive two-step procedure: First, the agent computes a personal reservation
price for the asset, given her current inventory. Second, she calibrates her bid and ask
quotes to the limit order book, by considering the probability with which her quotes will
be executed as a function of their distance from the midpoint price.

\section{Model assumptions}\label{sec:3.2}

The paper of \cite{AS2008} is closely related to that of \cite{HS1981}, with the 
crucial difference being that while Ho and Stoll consider a monopolistic dealer,
Avellaneda and Stoikov consider a dealer who is potentially one of many dealers 
and many other market participants who may set limit orders. 

In Ho and Stoll (1981),
the authors specify a `true' price for the asset, and then allow the dealer to set 
quotes around this price. This may be more applicable to OTC markets in illiquid 
products where there is no openly accessible limit orderbook, but Avellaneda and Stoikov
consider a dealer operating in an openly accessible limit orderbook, and hence it makes
sense to view the mid-point price in the orderbook as the true price of the security. 

Another point to make here is that Avellaneda and Stoikov consider a dealer who is 
concerned only with inventory risk, not asymmetric information, and so assuming that
other market participants are better informed and reacting to fundamental arbitrage
opportunities, by the efficient market hypothesis we would have that the market mid
price is the best available measure of the true price of the asset given all the information
available up to a particular point in time.

We will assume that the market mid-point price evolves according to the SDE
\begin{equation}
    \mathrm dS_u=\sigma\mathrm dW_u
\end{equation}
with initial value $S_t=s$. $W_t$ is a standard one-dimensional Brownian motion, and
$\sigma>0$ is constant. Underlying this model is an implicit assumption that the agent
has no opinion on the drift or any autocorrelation or stochasticity of volatility 
for the stock. 

We also assume for simplicity that the money market pays no interest. Moreover, the
limit orders set by the agent can be continuously updated at no cost. In reality, 
the cost of trading will differ depending on the exchange in question, as most
charge a small percentage fee of every executed trade and some only charge market orders,
while providing rebates to dealers' trades for the liquidity they provide. Finally,
we assume that the lot sizes our limit orders are constant at one share per order,
and that the overall arrival frequency of market orders is constant.

We summarise our assumptions in the list below:

\begin{itemize}
    \item The dealer being modelled is one of many players in the market
    \item The `true' price is given by the market mid-price
    \item The mid-price evolves according to a brownian motion with constant volatility $\sigma$
    \item The agent has no opinion on drift or autocorrelation of the stock price
    \item The money-market pays no interest and the agent can borrow with no interest
    \item Limit orders can be continuously updated at no cost
    \item Limit orders are of fixed size 1
    \item The arrival frequency of market orders to the market is constant
\end{itemize}

\section{Modelling an inactive trader}\label{sec:3.3}

Our agents objective will be to maximise the expected utility of their wealth at 
a terminal time $T$. Avellaneda and Stoikov's choice of exponential utility is 
convenient since its convexity allows us to define reservation prices that are 
indepenent of the agents current wealth.

\subsection*{The utility function}

Initially, we consider an inactive trader who holds a fixed inventory of $q$ stocks 
until the terminal time $T.$ The agent's value function is
\begin{equation}\label{eq:3.1}
    v(x,s,q,t)=\mathbb{E}\left[-e^{-\gamma(x+qS_T)}|\mathcal{F}_t\right]
\end{equation}
where $x$ is the initial wealth in dollars, $t$ is the present time and $\gamma$ is 
a personal pre-defined risk-aversion parameter. By some simple manipulations, we can 
write this in a more convenient form as follows:
\begin{align*}
    v(x,s,q,t)&=\mathbb{E}\left[-e^{-\gamma(x+qS_T)}|\mathcal{F}_t\right]\\
    &=-e^{-\gamma x}\mathbb{E}\left[e^{-\gamma q S_T}|\mathcal{F}_t\right]\\
    &=-e^{-\gamma x}e^{-\gamma q s + \frac{\gamma^2q^2\sigma^2(T-t)}{2}}\\
    &=-e^{-\gamma x}e^{-\gamma q s}e^{\frac{\gamma^2q^2\sigma^2(T-t)}{2}}
\end{align*}

\subsection*{Reservation prices}

Following \cite{AS2008}, we can now use our value function to define the agents 
reservation bid and ask prices.
The reservation bid and reservation ask prices are simply the prices at which the agent
is indifferent between buying/selling and doing nothing. In other words, the reservation
bid (ask) is the price at which the agent is indifferent between her current portfolio
and her current portfolio $\pm$ one stock and $\mp$ the cash price.

\begin{definition}[Reservation bid price]\label{def:3.3.1}
    Let $v$ be the value function of the agent. Its reservation bid price $r^b$ is
    given implicitly by the relation
    \begin{equation}\label{eq:3.2}
        v(x-r^b(s,q,t),s,q+1,t)=v(x,s,q,t)
    \end{equation}
    and the corresponding reservation ask price $r^a$ is similarly implicit in the 
    relation
    \begin{equation}\label{eq:3.3}
        v(x+r^a(s,q,t),s,q-1,t)=v(x,s,q,t).
    \end{equation}
\end{definition}

We can determine an exact expression for $r^b(s,q,t)$ by plugging our prior definition
for the value function, (\ref{eq:3.1}), in to our relation (\ref{eq:3.2}) as follows:
\begin{align*}
    v(x-r^b(s,q,t),s,q+1,t)&=v(x,s,q,t)\\
    -e^{-\gamma(x-r^b(s,q,t))}e^{-\gamma s(q+1)}e^{\frac{\gamma^2(q+1)^2\sigma^2(T-t)}{2}}&=-e^{-\gamma x}e^{-\gamma q s}e^{\frac{\gamma^2q^2\sigma^2(T-t)}{2}}\\
    -\gamma(x-r^b(s,q,t))-\gamma s(q+1) + \frac{\gamma^2(q+1)^2\sigma^2(T-t)}{2} &= -\gamma x-\gamma q s + \frac{\gamma^2q^2\sigma^2(T-t)}{2}\\
    \gamma r^b(s,q,t)-\gamma s + \frac{\gamma^2(1+2q)\sigma^2(T-t)}{2} &=0,
\end{align*}

dividing by $\gamma$ and rearranging to obtain
\begin{equation}
    r^b(s,q,t)=s+(-1-2q)\frac{\gamma\sigma^2(T-t)}{2}
\end{equation}
Similarly for $r^a(s,q,t)$:
\begin{align*}
    v(x+r^a(s,q,t),s,q-1,t)&=v(x,s,q,t)\\
    -e^{-\gamma(x+r^a(s,q,t))}e^{-\gamma s(q-1)}e^{\frac{\gamma^2(q-1)^2\sigma^2(T-t)}{2}}&=-e^{-\gamma x}e^{-\gamma q s}e^{\frac{\gamma^2q^2\sigma^2(T-t)}{2}}\\
    -\gamma(x+r^a(s,q,t))-\gamma s(q-1)+\frac{\gamma^2(q-1)^2\sigma^2(T-t)}{2}&=-\gamma x-\gamma q s + \frac{\gamma^2q^2\sigma^2(T-t)}{2}\\
    -\gamma r^a(s,q,t) + \gamma s + \frac{\gamma^2(1-2q)\sigma^2(T-t)}{2}&=0,
\end{align*}
again dividing by $\gamma$ and rearranging to obtain
\begin{equation}
    r^a(s,q,t)=s+(1-2q)\frac{\gamma\sigma^2(T-t)}{2}
\end{equation}
We define the \emph{reservation} or \emph{indifference} price to be the average of 
these two \textit{given} that the agent currently holds q stocks:
\begin{align*}
    r(s,q,t)&=\frac{r^a(s,q,t)+r^b(s,q,t)}{2}\\
    &=\frac{s+(1-2q)\frac{\gamma\sigma^2(T-t)}{2}+s+(-1-2q)\frac{\gamma\sigma^2(T-t)}{2}}{2}\\
    &=\frac{2s-2q\gamma\sigma^2(T-t)}{2}\\
    &=s-q\gamma\sigma^2(T-t)
\end{align*}
This price is nothing more than an adjustment to the mid-price which accounts for the 
effect of the inventory held by the agent on the agents preference to buy or sell. It
is easy to see that if the agent is long stock ($q>0$), the reservation price will be 
lower than the mid-price, reflecting the agents willingness to sell at a discount in 
order to reduce its inventory. Conversely, if the agent is short stock ($q<0$), its 
reservation price will be greater than the mid-price, indicating the agents preference
to buy at a premium to the market in order to return to a market-neutral position.

We note that the expressions derived above for $r^a$ and $r^b$ (and consequently $r$)
exist in the setting where $q$ is a fixed constant, and therefore it is not so simple
to derive these expressions when our agent is permitted to set limit orders. However,
they are important both as an illustrative example and because when we introduce our 
approximate solution in \ref{sec:3.8}, we will arrive at a very similar reservation price.

\section{The Optimising Agent with Infinite Horizon}\label{sec:3.4}
We will now briefly analyse the infinite horizon variant of the dealer problem, showing
that we can derive a stationary version of the reservation price through defining
an infinite horizon variant of our value function including a discount factor. This is
necessary since in our finite horizon case discussed above, our reservation price is
dependent upon the time interval $T-t.$ The intuition for this is that at or close to 
$T$, the agent may liquidate any remaining inventory for (or at least close to) $S_T$,
hence the closer time is to $T$, the less risk there is in the dealer's position.

We consider an infinite-horizon value function of the form
\begin{equation*}
    \bar v(x,s,q)=\mathbb{E}\left[\int_{0}^{\infty}-e^{-\omega t}e^{-\gamma(x+qS_t)}\mathrm dt\right]
\end{equation*}
where $\omega$ is our discount factor. An interpretation of $\omega$ is that it
represents an upper bound on the absolute inventory position that the agent is allowed
to build up. A natural choice is to take $\omega=\frac{1}{2}\gamma^2\sigma^2(q_{\textrm{max}}+1)^2$
, this will be justified shortly. 

Using the definition of reservation bid and ask 
prices given above in section \ref{sec:3.3}, we can attain stationary versions of the 
reservation prices $r^b$ and $r^a$ with much the same method as before, only relying on
slightly more advanced theory, appealing to Tonelli's theorem (\ref{eq:1.13}) which 
allows us to swap the expectation and integral in the value function. For $r^b$, we 
have the following:
\begin{align*}
    \bar{v}(x-\bar{r}^b(s,q),s,q+1)&=\bar{v}(x,s,q)\\
    \mathbb{E}\left[\int_{0}^{\infty}-e^{-\omega t}e^{-\gamma(x-\bar{r}^b(s,q)+(q+1)S_t)}\mathrm dt\right]&=\mathbb{E}\left[\int_{0}^{\infty}-e^{-\omega t}e^{-\gamma(x+qS_t)}\mathrm dt\right]\\
    \int_{0}^{\infty}e^{-\omega t}e^{-\gamma(x-\bar{r}^b(s,q))}\mathbb{E}\left[e^{-\gamma(q+1)S_t}\right]\mathrm dt&=\int_{0}^{\infty}e^{-\omega t}e^{-\gamma x}\mathbb{E}\left[e^{-\gamma qS_t}\right]\mathrm dt\textrm{ (by Tonelli)}\\
    e^{-\gamma(x-\bar{r}^b(s,q))}\int_{0}^{\infty}e^{-\omega t}e^{-\gamma(q+1)s+\frac{\gamma^2(q-1)^2\sigma^2t}{2}}\mathrm dt&=e^{-\gamma x}\int_{0}^{\infty}e^{-\omega t}e^{-\gamma qs+\frac{\gamma^2q^2\sigma^2t}{2}}\mathrm dt\\
    e^{-\gamma(x-\bar{r}^b(s,q))}e^{-\gamma(q+1)s}\int_{0}^{\infty}e^{-\omega t}e^{\frac{\gamma^2(q+1)^2\sigma^2t}{2}}\mathrm dt&=e^{-\gamma x}e^{-\gamma qs}\int_{0}^{\infty}e^{-\omega t}e^{\frac{\gamma^2q^2\sigma^2t}{2}}\mathrm dt\\
    e^{\gamma \bar{r}^b(s,q)}e^{-\gamma s}\int_{0}^{\infty}e^{\left(\frac{\gamma^2(q+1)^2\sigma^2-2\omega}{2}\right)t}\mathrm dt&=\int_{0}^{\infty}e^{\left(\frac{\gamma^2q^2\sigma^2-2\omega}{2}\right)t}\mathrm dt\\
    e^{\gamma \bar{r}^b(s,q)}e^{-\gamma s}\left(\frac{2}{2\omega-\gamma^2(q+1)^2\sigma^2}\right)&=\left(\frac{2}{2\omega-\gamma^2q^2\sigma^2}\right)\\
    e^{\gamma(\bar{r}^b(s,q)-s)}&=\frac{2\omega-\gamma^2(q+1)^2\sigma^2}{2\omega-\gamma^2q^2\sigma^2}\\
    e^{\gamma(\bar{r}^b(s,q)-s)}&=1-\frac{(1+2q)\gamma^2\sigma^2}{2\omega-\gamma^2q^2\sigma^2}\\
    \gamma\bar{r}^b(s,q)-\gamma s&=\log\left(1+\frac{(-1-2q)\gamma^2\sigma^2}{2\omega-\gamma^2q^2\sigma^2}\right)\\
    \bar{r}^b(s,q) &= s+\frac{1}{\gamma}\log\left(1+\frac{(-1-2q)\gamma^2\sigma^2}{2\omega-\gamma^2q^2\sigma^2}\right)
\end{align*}
which is valid for $\omega>\frac{1}{2}\gamma^2\sigma^2q^2$ and agrees exactly with
the result presented in \cite{AS2008}. We can now perform the same procedure for the
reservation ask price $r^a$:
\begin{align*}
    \bar{v}(x+r^a(s,q),s,q-1)&=\bar{v}(x,s,q)\\
    \mathbb{E}\left[\int_{0}^{\infty}-e^{-\omega t}e^{-\gamma(x+r^a(s,q)+(q-1)S_t)}\mathrm dt\right]&=\mathbb{E}\left[\int_{0}^{\infty}-e^{-\omega t}e^{-\gamma(x+qS_t)}\mathrm dt\right]\\
    \int_{0}^{\infty}e^{-\omega t}e^{-\gamma(x+r^a(s,q))}\mathbb{E}\left[e^{-\gamma(q-1)S_t}\right]\mathrm dt&=\int_{0}^{\infty}e^{-\omega t}e^{-\gamma x}\mathbb{E}\left[e^{-\gamma qS_t}\right]\mathrm dt\textrm{ (by Tonelli)}\\
    e^{-\gamma(x+r^a(s,q))}\int_{0}^{\infty}e^{-\omega t}e^{-\gamma(q-1)s+\frac{\gamma^2(q-1)^2\sigma^2t}{2}}\mathrm dt&=e^{-\gamma x}\int_{0}^{\infty}e^{-\omega t}e^{-\gamma qs+\frac{\gamma^2q^2\sigma^2t}{2}}\mathrm dt\\
    e^{-\gamma(x+r^a(s,q))}e^{-\gamma(q-1)s}\int_{0}^{\infty}e^{-\omega t}e^{\frac{\gamma^2(q-1)^2\sigma^2t}{2}}\mathrm dt&=e^{-\gamma x}e^{-\gamma qs}\int_{0}^{\infty}e^{-\omega t}e^{\frac{\gamma^2q^2\sigma^2t}{2}}\mathrm dt\\
    e^{-\gamma r^a(s,q)}e^{\gamma s}\int_{0}^{\infty}e^{\left(\frac{\gamma^2(q-1)^2\sigma^2-2\omega}{2}\right)t}\mathrm dt&=\int_{0}^{\infty}e^{\left(\frac{\gamma^2q^2\sigma^2-2\omega}{2}\right)t}\mathrm dt\\
    e^{-\gamma r^a(s,q)}e^{\gamma s}\left(\frac{2}{2\omega-\gamma^2(q-1)^2\sigma^2}\right)&=\left(\frac{2}{2\omega-\gamma^2q^2\sigma^2}\right)\\
    e^{\gamma(s-r^a(s,q))}&=\frac{2\omega-\gamma^2(q-1)^2\sigma^2}{2\omega-\gamma^2q^2\sigma^2}\\
    e^{\gamma(s-r^a(s,q))}&=1-\frac{(1-2q)\gamma^2\sigma^2}{2\omega-\gamma^2q^2\sigma^2}\\
    \gamma s - \gamma r^a(s,q)&=\log\left(1-\frac{(1-2q)\gamma^2\sigma^2}{2\omega-\gamma^2q^2\sigma^2}\right)\\
    r^a(s,q) &= s-\frac{1}{\gamma}\log\left(1-\frac{(1-2q)\gamma^2\sigma^2}{2\omega-\gamma^2q^2\sigma^2}\right)
\end{align*}
which is again valid for $\omega>\frac{1}{2}\gamma^2\sigma^2q^2$. However, this differs
from the result presented in \cite{AS2008}, which they give to be:
\begin{equation*}
    r^a(s,q)=s+\frac{1}{\gamma}\log\left(1+\frac{(1-2q)\gamma^2\sigma^2}{2\omega-\gamma^2q^2\sigma^2}\right)
\end{equation*}
From our derivations, we can see that to ensure integrability in both cases, the 
user-specified parameter $\omega$ must satisfy
\begin{equation*}
    \omega>\frac{1}{2}\gamma^2\sigma^2q^2,
\end{equation*}
where the only variable quantity on the RHS is the inventory variable $q$. Therefore,
if we want to ensure that these reserve prices always exist, we should bound the maximum
inventory our agent can build up on either side. Hence, we set some $q_{\textrm{max}}>0$
and set
\begin{equation*}
    \omega=\frac{1}{2}\gamma^2\sigma^2(q_{\textrm{max}}+1)^2
\end{equation*}
to ensure integrability.

\section{Modelling Limit Orders}\label{sec:3.5}
Now that we have defined and discussed the idea of a reservation price
for the dealer, we should move on to considering the case of the dealer
who can indirectly influence both their inventory and cash flow through
the setting of limit orders. 

As mentioned in section \ref{sec:3.2},
the agent quotes bid and ask limit orders in lot sizes of 1 only.
We denote the agent's quotes $p^a$ and $p^b$ for the ask and bid respectively,
and note that the agent is comitted to sell or buy 1 unit of stock respectively
should these orders be ``hit'' or ``lifted'' by an incoming market order.
These quotes can also be updated continuously at no cost. The distances
\begin{equation}
    \delta^a:=p^a-s
\end{equation}
and 
\begin{equation}
    \delta^b:=s-p^b
\end{equation}
as well as the current shape of the orderbook determine the priority 
of execution when large market orders are placed. 

For example, when a market order to buy $Q$ shares arrives, the $Q$ 
limit orders with the lowest ask prices will be lifted automatically
by the exchanges matching engine. If $Q$ is greater than the number
of shares available at the lowest ask level in the orderbook, the order
causes a temporary market impact since transactions will occur at a 
price not only higher than the mid-price, but higher than the best ask.

\begin{definition}[Temporary market impact]
    Let $p^Q$ be the price of the highest (most expensive) limit order 
    executed in this trade. Then 
    \begin{equation}
        \Delta p:=p^Q-s
    \end{equation}
    is the temporary market impact of the trade of size $Q.$ 
\end{definition}

Then we have that if our agent's $\delta^a < \Delta p$, our agents 
limit order will be executed. We will assume that market buy orders
will lift our agent's sell limit orders with a Poisson intensity function
denoted $\lambda^a(\delta^a)$ which is a decreasing function of $\delta^a$.
Likewise, we assume that market sell orders will hit our agent's bid 
limit orders with Poisson intensity $\lambda^b(\delta^b)$, decreasing
in $\delta^b$. Intuitively, this encapsulates the fact that further 
away from the mid-price the agent places her quotes, the less often she
will receive market orders.

Now, our cash wealth and portfolio of stock is stochastic and depends
on the incoming flow of market buy and sell orders. Naturally, both
our cash flow and inventory jump every time a market order executes 
one of our agent's limit orders. Let $N^a_t$ and $N^b_t$ be Poisson 
point processes with intensities $\lambda^a$ and $\lambda^b$, 
representing the amount of stocks sold or bought by the agent up to 
time t respectively. Our inventory at time t is thus 
\begin{equation}
    q_t:=N^b_t-N^a_t
\end{equation} 
and our wealth process evolves according to
\begin{equation}
    \mathrm dXt=p^a\mathrm dN^a_t-p^b\mathrm dN^b_t.
\end{equation}
Finally, we can reformulate our value function from section \ref{sec:3.3}.
The goal we set for our agent is still to maximise the expected 
expontential utility of terminal wealth, however now the cash and 
inventory components of our terminal portfolio are stochastic as 
well as the mid-price itself. Hence, our value function becomes 
the following:
\begin{definition}[Value function of Market-Making Agent]
    \begin{equation}\label{eq:3.12}
        u(s,x,q,t):=\max\limits_{\delta^a,\delta^b}\mathbb{E}\left[-e^{-\gamma(X_T+q_TS_T)}|\mathcal{F}_t\right]
    \end{equation}
\end{definition}

Notice that our agent choses its quote spreads $\delta^a$ and $\delta^b$,
and hence controls its quotes $p^a$ and $p^b$. This means that the 
agent therefore indirectly influences the flow of orders she receives.

In the next section we will consider some realistic forms for the functions
$\lambda^a$ and $\lambda^b$ based on results in the econophysics literature
exploring the statistical properties of the limit orderbook, before 
turning to the application of Stochastic Control and solution to the 
above problem in section \ref{sec:3.7}.

\section{Modelling Trading Intensity}\label{sec:3.6}

Here, we will focus on deriving a realistic form for the Poisson intensity
$\lambda$ with which a limit order will be executed as a function of its
distance $\delta$ to the mid-price. In order to quantify this, we need 
to infer some statistics regarding
\begin{itemize}
    \item The overall frequency of market orders
    \item The distribution of the size of market orders
    \item The temporary price impact of a large market order
\end{itemize}
For simplicity, we will assume a constant frequency $\Lambda$ of market
buy or sell orders. In practice, this could be estimated by simply 
dividing the total volume bought or sold in a given time interval 
by the average volume of market buy/sell orders in that interval. 

\subsection*{Distribution of the size of market orders}

The distribution of size of market orders has been found to obey a 
power law:
\begin{theorem}[Density of Market Order Size]
    The distribution of size of market orders has been found to obey a 
    power law:
    \begin{equation}\label{eq:3.13}
        f^{Q}(x)\propto x^{-1-\alpha}
    \end{equation}
    for large $x$, with $\alpha=1.53$ in \cite{Gopi2000} for US stocks,
    $\alpha=1.4$ in \cite{Maslov2001} for shares traded on the NASDAQ
    and $\alpha=1.5$ in \cite{Gabaix2006} for shares on the Paris Bourse.
\end{theorem}

\subsection*{Modelling market impact}

Here there is much less consensus on market impact, due to lack of
agreement on how to define it and how to measure it. Some papers
find that the change in price $\Delta p$ after a market order of size 
$Q$ is described well by 
\begin{equation}\label{eq:3.14}
    \Delta p\propto Q^\beta
\end{equation}
with $\beta=0.5$ in \cite{Gabaix2006} and $\beta=0.76$ in \cite{Weber2005},
while \cite{Potters2003} find a better fit to the relationship
\begin{equation}\label{eq:3.15}
    \Delta p\propto\log(Q).
\end{equation}
Using (\ref{eq:3.13}) and (\ref{eq:3.15}) we can derive the poisson 
intensity as follows:
\begin{align*}
    \lambda(\delta)&=\Lambda\mathbb{P}(\delta<\Delta p)\\
    &=\Lambda\mathbb{P}\left(\delta<\frac{\log Q}{K}\right)\\
    &=\Lambda\mathbb{P}(K\delta<\log Q)\\
    &=\Lambda\mathbb{P}\left(e^{K\delta}<Q\right)\\
    &=\Lambda\int_{e^{K\delta}}^{\infty}x^{-1-\alpha}dx\\
    &=\Lambda\left[\frac{-x^{-\alpha}}{\alpha}\right]_{e^{K\delta}}^\infty\\
    &=\Lambda\left(\lim_{t\rightarrow\infty}\frac{-t^{-\alpha}}{\alpha}+\frac{e^{-K\delta\alpha}}{\alpha}\right)\\
    &=\frac{\Lambda}{\alpha}\left(e^{-K\delta\alpha}-\lim_{t\rightarrow\infty}\frac{1}{t^\alpha}\right)\\
    &=\frac{\Lambda}{\alpha}e^{-\alpha K\delta}\\
    &=Ae^{-k\delta}
\end{align*}
where $A=\frac{\Lambda}{\alpha}$ and $k=\alpha K$. 
On the other hand, (\ref{eq:3.13}) and (\ref{eq:3.14}) yield:
\begin{align*}
    \lambda(\delta)&=\Lambda\mathbb{P}(\delta<\Delta p)\\
    &=\Lambda\mathbb{P}(\delta<kQ^\beta)\\
    &=\Lambda\mathbb{P}\left(Q>\left(\frac{\delta}{k}\right)^{-\beta}\right)\\
    &=\Lambda\int_{\left(\frac{\delta}{k}\right)^{-\beta}}^\infty x^{-1-\alpha}dx\\
    &=\Lambda\left[\lim_{t\rightarrow\infty}\frac{-t^{-\alpha}}{\alpha}+\frac{\left(\frac{\delta}{k}\right)^{-\frac{\alpha}{\beta}}}{\alpha}\right]\\
    &=\frac{\Lambda\left(\frac{\delta}{k}\right)^{-\frac{\alpha}{\beta}}}{\alpha}\\
    &=B\delta^{-\frac{\alpha}{\beta}}
\end{align*}
where $B=\frac{\Lambda}{k\alpha}$.
Alternatively, we could derive the price impact function $\Delta p$ directly by 
integrating the density of the orderbook as described in \cite{Weber2005}
and \cite{Smith2003}. However, this leads to a function which is highly dependent
on the orderbook in question, so in the interest of obtaining a more general result
in the following sections, we will not cover this method here.
%\textbf{Include discussion and explanation of this method and derived form of $\lambda$}

\newpage
\section{The Hamilton-Jacobi-Bellman Equation}\label{sec:3.7}

Now that we have formulated our agent's value function, and discussed 
some empirical results on the form of the Poisson intensity $\lambda$,
we turn to the solution of the problem at hand. Following on from our 
discussion of the theory of stochastic control in Chapter \ref{chap:2},
our first goal will be to formulate the Hamilton-Jacobi-Bellman PDE
associated to our value function which we defined in (\ref{eq:3.12}).
Recall that this is given by
\begin{equation}
    v(s,x,q,t)=\max\limits_{\delta^a,\delta^b}\mathbb{E}\left[-e^{-\gamma(X_T+q_TS_T)}|\mathcal{F}_t\right]
\end{equation}
where our optimal control processes $\delta^a$ and $\delta^b$ will
turn out to be time and state dependent. This type of optimal dealer 
problem was first studied by \cite{HS1981}, who use the Dynamic Programming 
Principle to show that $v$ satisfies the following HJB:
\begin{theorem}[Ho and Stoll (1981) HJB]
    \begin{equation}\label{eq:3.17}
        \left\{
            \begin{aligned}
                v_t+\frac{1}{2}\sigma^2v_{ss}&+\max\limits_{\delta^b}\lambda^b(\delta^b)[v(s,x-s+\delta^b,q+1,t)-v(s,x,q,t)]\\
                &+\max\limits_{\delta^a}\lambda^a(\delta^a)[v(s,x+s+\delta^a,q-1,t)-v(s,x,q,t)]=0,\\
                &v(s,x,q,T)=-e^{-\gamma(x+qs)}.
            \end{aligned}
        \right.
    \end{equation}
\end{theorem}
\begin{proof}
    Recall from chapter \ref{chap:2} that the value function in general follows the 
    form given in (\ref{eq:2.19}) and shown below:
    \begin{equation*}
        \left\{
        \begin{aligned}
            &\frac{\partial v}{\partial t}(t,x)+\sup_{a\in A}\left[\mathcal{L}^av(t,x)+f(t,x,a)\right]=0\;\forall\;(t,x)\in[0,T)\times\mathbb{R}\\
            &v(T,x)=g(x)\;\forall\;x\in\mathbb{R}.
        \end{aligned}
        \right.
    \end{equation*}
    First, we will check our boundary condition. At time $T$, the trading period 
    has ended, and so in (\ref{eq:3.12}) we can ignore the maximisation over 
    controles $\delta^a$ and $\delta^b$ when evaluating $v$ at $T$. Moreover,
    $X_T$, $q_T$ and $S_T$ are all adapted and hence measurable with respect to 
    $\mathcal{F}_T$, and we can remove the expectation. Hence we find that 
    \begin{equation}
        v(s,x,q,T)=-e^{-\gamma(X_T+q_TS_T)}
    \end{equation}
    as expected. Next, for the dynamics of the system, note that the operator 
    $\mathcal{L}^\alpha v$ defined as 
    \begin{align*}
        \mathcal{L}^\alpha v(t,x)&=b(t,x,\alpha)v_x+\frac{1}{2}\sigma(t,x,\alpha)^2v_{xx}\\
        &=\frac{1}{2}\sigma(t,x,\alpha)^2v_{xx}\textrm{ since we assumed a brownian motion without drift}\\
        &=\frac{1}{2}\sigma^2v_{xx}\textrm{ since we assumed constant volatility $\sigma$}
    \end{align*}
    and we also notice that this has no dependence on our control process 
    $\alpha=\begin{pmatrix}\delta^{a}\\\delta^{b}\end{pmatrix}$.

    For the incremental gains encapuslated by the function $f$, we notice that 
    by the properties of the Poisson process, the density of ask or bid orders
    arriving at time $t$ is $\lambda^a(\delta^a)$ or $\lambda^b(\delta^b)$ respectively.
    Hence the density of changes to the value function over infinitessimal units of 
    time is given by 
    \begin{equation*}
        \lambda^b(\delta^b)\times(\textrm{increment to $v$ caused by 1 bid order})=\lambda^b(\delta^b)[v(s,x-s+\delta^b,q+1,t)-v(s,x,q,t)]
    \end{equation*}
    for market sell orders hitting our bid orders, and the corresponding expression 
    for market buy orders lifting our ask orders is 
    \begin{equation*}
        \lambda^a(\delta^a)[v(s,x+s+\delta^a,q-1,t)-v(s,x,q,t)].
    \end{equation*}
    Putting this all together, we have that 
    \begin{equation*}
        \frac{\partial v}{\partial t}(t,x)+\sup_{a\in A}\left[\mathcal{L}^av(t,x)+f(t,x,a)\right]=0
    \end{equation*}
    equates to 
    \begin{equation*}
        \left\{
            \begin{aligned}
                v_t+\frac{1}{2}\sigma^2v_{ss}&+\max\limits_{\delta^b}\lambda^b(\delta^b)[v(s,x-s+\delta^b,q+1,t)-v(s,x,q,t)]\\
                &+\max\limits_{\delta^a}\lambda^a(\delta^a)[v(s,x+s+\delta^a,q-1,t)-v(s,x,q,t)]=0
            \end{aligned}
        \right.
    \end{equation*}
    for our particular value function $v$. 
\end{proof}

\begin{remark}
    \cite{AS2008} throughout their paper refer to maxima, rather than suprema, over
    the value functions. This is justified because although the optimal quotes 
    are real-valued, modern electronic markets typically trade in integer amounts 
    of fractions of pennies, restricting dealers to a countable set of possible quotes.
    Optimal quotes can simply be rounded to the nearest attainable quote in order to 
    be used for trading.
\end{remark}

In order to simplify the problem before turning to look at its solution, 
\cite{AS2008} argue that due to our choice of exponential 
utility, we can introduce the following ansatz:
\begin{equation}\label{eq:3.18}
    v(s,x,q,t)=-e^{-\gamma x}e^{-\gamma\theta(s,q,t)}
\end{equation}
The intuition behind this is that considering our value function at 
time $t$, our current wealth $X_t$ is a predetermined constant and thus
measurable w.r.t $\mathcal{F}_t$. Hence we can take $-e^{-\gamma x}$
out from the expectation. The remainder, being our future cash flow, 
future inventory and terminal portfolio value are all time and state
dependent and hence encapsulated by some function $\theta$ of $s$, $q$
and $t$. Moreover, thanks to the properties of the exponential function, 
the expectation of the utility of our future wealth can also be 
written in an exponential form. Finally, we also assume that the function
$\theta$ factors in our optimal control $\alpha^*=\begin{pmatrix}\delta^{a*}\\\delta^{b*}\end{pmatrix}$

By substitution of Avellaneda and Stoikov's ansatz (\ref{eq:3.18}) into 
Ho and Stoll's HJB (\ref{eq:3.17}) we obtain the following HJB equation
for $\theta$:
\begin{theorem}[Avellaneda and Stoikov (2008) HJB]\label{thm:3.7.2}
    \begin{equation}\label{eq:3.19}
        \left\{
            \begin{aligned}
                \theta_t+\frac{1}{2}\sigma^2\theta_{ss}-\frac{1}{2}\sigma^2\gamma\theta_{s}^2&+\max\limits_{\delta^b}\left[\frac{\lambda^b(\delta^b)}{\gamma}(1-e^{\gamma(s-\delta^b-r^b)})\right]\\
                &+\max\limits_{\delta^a}\left[\frac{\lambda^a(\delta^a)}{\gamma}(1-e^{-\gamma(s+\delta^a-r^a)})\right]=0,\\
                &\theta(s,q,T)=qs.
            \end{aligned}
        \right.
    \end{equation}
\end{theorem}

\subsection*{Relations for the reserve prices}

Before we can prove that this subsitution provides an equivalent formulation of our
Hamilton-Jacobi-Bellman equation, we need a lemma relating the definitions we gave
of the dealer's reservation bid and ask prices in section \ref{sec:3.3} to our new
function $\theta$. We find that we can express $r^b$ and $r^a$ directly in terms of 
$\theta$ as follows:
\begin{lemma}\label{lem:3.7.3}
    We have using the ansatz \ref{eq:3.18} that the reservation bid and ask prices
    defined in definition \ref{def:3.3.1} are given by
    \begin{equation}
        r^b(s,q,t)=\theta(s,q+1,t)-\theta(s,q,t)
    \end{equation}
    and
    \begin{equation}
        r^a(s,q,t)=\theta(s,q,t)-\theta(s,q-1,t).
    \end{equation}
\end{lemma}
\begin{proof}
    We prove the above directly from the definition of the reserve bid and ask 
    respectively:
    \begin{align*}
        v(s,x-r^b(s,q,t),q+1,t)&=v(s,x,q,t)\\
        -e^{-\gamma(x-r^b(s,q,t))}e^{-\gamma\theta(s,q+1,t)}&=-e^{-\gamma x}e^{-\gamma\theta(s,q,t)}\\
        -\gamma(x-r^b(s,q,t))-\gamma\theta(s,q+1,t)&=-\gamma x-\gamma\theta(s,q,t)\\
        x-r^b(s,q,t)+\theta(s,q+1,t)&=x+\theta(s,q,t)\\
        r^b(s,q,t)&=\theta(s,q+1,t)-\theta(s,q,t)
    \end{align*}
    and
    \begin{align*}
        v(s,x+r^a(s,q,t),q-1,t)&=v(s,x,q,t)\\
        -e^{-\gamma(x+r^a(s,q,t))}e^{-\gamma\theta(s,q-1,t)}&=-e^{-\gamma x}e^{-\gamma\theta(s,q,t)}\\
        -\gamma(x+r^a(s,q,t))-\gamma\theta(s,q-1,t)&=-\gamma x-\gamma\theta(s,q,t)\\
        x+r^a(s,q,t)+\theta(s,q-1,t)&=x+\gamma\theta(s,q,t)\\
        r^a(s,q,t)&=\theta(s,q,t)-\theta(s,q-1,t).
    \end{align*}
\end{proof}

Using this result, we can check that the ansatz \ref{eq:3.18} allows us to derive 
the HJB equation given in theorem \ref{thm:3.7.2}.

\begin{proof}[Proof of Theorem \ref{thm:3.7.2}]
    First we check the terminal condition:
    \begin{align*}
        v(s,x,q,T)&=-e^{-\gamma(x+qs)}\textrm{ from (\ref{eq:3.17})}\\
        &=-e^{-\gamma x}e^{-\gamma\theta(s,q,T)}\textrm{ from (\ref{eq:3.18})}\\
        &=-e^{-\gamma(x+\theta(s,q,T))}\\
        \implies \theta(s,q,T)&=qs.
    \end{align*}
    which is what we expected. Next, by direct substitution, note that
    \begin{equation*}
        v_t = -\frac{\partial}{\partial t}e^{-\gamma x}e^{-\gamma\theta}=-e^{-\gamma x}\times-\gamma\theta_te^{-\gamma\theta}=\gamma e^{-\gamma x}\theta_te^{-\gamma\theta}
    \end{equation*}
    and 
    \begin{align*}
        \frac{1}{2}\sigma^2v_{ss}&=-e^{-\gamma x}\frac{1}{2}\sigma^2\frac{\partial^2}{\partial s^2}e^{-\gamma\theta}\\
        &=\gamma e^{-\gamma x}\frac{1}{2}\sigma^2\frac{\partial}{\partial s}\theta_se^{-\gamma\theta}\\
        &=\gamma e^{-\gamma x}\frac{1}{2}\sigma^2\left(\theta_{ss}e^{-\gamma\theta}-\gamma\theta_s^2e^{-\gamma\theta}\right).
    \end{align*}
    Next we consider the maximised terms:
    \begin{align*}
        \lambda^b(\delta^b)[v(s,x-s+\delta^b,q+1,t)-v(s,x,q,t)]&=\lambda^b(\delta^b)[-e^{-\gamma (x-s+\delta^b)}e^{-\gamma\theta(s,q+1,t)}+e^{-\gamma x}e^{-\gamma\theta(s,q,t)}]\\
        &=\lambda^b(\delta^b)[e^{-\gamma x}e^{-\gamma\theta(s,q,t)}-e^{-\gamma x}e^{\gamma s}e^{-\gamma \delta^b}e^{-\gamma\theta(s,q+1,t)}]
    \end{align*}
    and
    \begin{align*}
        \lambda^a(\delta^a)[v(s,x+s+\delta^a,q-1,t)-v(s,x,q,t)]&=\lambda^a(\delta^a)[-e^{-\gamma (x+s+\delta^a)}e^{-\gamma\theta(s,q-1,t)}+e^{-\gamma x}e^{-\gamma\theta(s,q,t)}]\\
        &=\lambda^a(\delta^a)[e^{-\gamma x}e^{-\gamma\theta(s,q,t)}-e^{-\gamma x}e^{-\gamma s}e^{-\gamma\delta^a}e^{-\gamma\theta(s,q-1,t)}].
    \end{align*}
    We note that since the R.H.S. of our equation is 0 and all of our 
    expressions contain $e^{-\gamma x}$, we can mutliply by this term.
    We can remove all $e^{-\gamma \theta}$ terms similarly. Dividing 
    all expressions by $\gamma$ and substituting into (\ref{eq:3.17})
    yields a L.H.S. of
    \begin{align*}
        \theta_t+\frac{1}{2}\sigma^2\theta_{ss}-\frac{1}{2}\sigma^2\gamma\theta_s^2&+\max\limits_{\delta^b}\frac{\lambda^b(\delta^b)}{\gamma}[1-e^{\gamma s}e^{-\gamma \delta^b}e^{-\gamma(\theta(s,q+1,t)-\theta(s,q,t))}]\\
        &+\max\limits_{\delta^a}\frac{\lambda^a(\delta^a)}{\gamma}[1-e^{-\gamma s}e^{-\gamma\delta^a}e^{\gamma(\theta(s,q,t)-\theta(s,q-1,t))}]
    \end{align*}
    which by lemma \ref{lem:3.7.3} simplifies to
    \begin{align*}
        \theta_t+\frac{1}{2}\sigma^2\theta_{ss}-\frac{1}{2}\sigma^2\gamma\theta_s^2&+\max\limits_{\delta^b}\frac{\lambda^b(\delta^b)}{\gamma}[1-e^{\gamma s}e^{-\gamma \delta^b}e^{-\gamma r^b(s,q,t)}]\\
        &+\max\limits_{\delta^a}\frac{\lambda^a(\delta^a)}{\gamma}[1-e^{-\gamma s}e^{-\gamma\delta^a}e^{\gamma r^a(s,q,t)}]\\
        =\theta_t+\frac{1}{2}\sigma^2\theta_{ss}-\frac{1}{2}\sigma^2\gamma\theta_s^2&+\max\limits_{\delta^b}\frac{\lambda^b(\delta^b)}{\gamma}[1-e^{\gamma (s-\delta^b-r^b(s,q,t))}]\\
        &+\max\limits_{\delta^a}\frac{\lambda^a(\delta^a)}{\gamma}[1-e^{-\gamma (s+\delta^a-r^a(s,q,t))}]
    \end{align*}
\end{proof}

\subsection*{Implicit relations for the optimal spreads $\delta^a$ and $\delta^b$}

We can derive some relations for the optimal distances $\delta^a$ and $\delta^b$
that are implicit in our slightly simplified HJB equation (\ref{eq:3.19}). Inspecting
the maximised terms in the HJB, we can invoke a first-order optimality condition to
find an expression involving the optimal spreads, the reservation prices, and our 
Poisson intensity $\lambda$. 
\begin{theorem}[Implicit relations for the optimal spreads $\delta^a$ and $\delta^b$]
    For $\delta^b$, we obtain the following:
    \begin{equation}\label{eq:3.22}
        s-r^b(s,q,t)=\delta^b-\frac{1}{\gamma}\log\left(1-\gamma\frac{\lambda^b(\delta^b)}{\frac{\partial\lambda^b}{\partial\delta}(\delta^b)}\right)
    \end{equation}
    while for $\delta^a$ we have
    \begin{equation}\label{eq:3.23}
        r^a(s,q,t)-s=\delta^a-\frac{1}{\gamma}\log\left(1-\gamma\frac{\lambda^a(\delta^a)}{\frac{\partial\lambda^a}{\partial\delta}(\delta^a)}\right)
    \end{equation}
\end{theorem}
\begin{proof}
    Taking the derivative of the term in \ref{eq:3.19} that we maximise w.r.t. $\delta^b$
    and setting it equal to 0, we find:
    \begin{align*}
        \frac{\partial}{\partial\delta}\left[\frac{\lambda^b(\delta)}{\gamma}(1-e^{\gamma(s-\delta-r^b(s,q,t))})\right](\delta^b)&=0\\
        \frac{1}{\gamma}\left[\frac{\partial\lambda^b}{\partial\delta}(\delta^b)-\frac{\partial}{\partial\delta}\lambda^b(\delta^b)e^{\gamma(s-\delta^b-r^b(s,q,t))}\right]&=0\\
        \frac{\partial\lambda^b}{\partial\delta}(\delta^b)-\frac{\partial\lambda^b}{\partial\delta}(\delta^b)e^{\gamma(s-\delta^b-r^b(s,q,t))}+\gamma\lambda^b(\delta^b)e^{\gamma(s-\delta^b-r^b(s,q,t))}&=0\\
        \left(\gamma\lambda^b(\delta^b)-\frac{\partial\lambda^b}{\partial\delta}(\delta^b)\right)e^{\gamma(s-\delta^b-r^b(s,q,t))}&=-\frac{\partial\lambda^b}{\partial\delta}(\delta^b)\\
        -\left(\frac{\partial\lambda^b}{\partial\delta}(\delta^b)\right)e^{-\gamma(s-\delta^b-r^b(s,q,t))}&=\gamma\lambda^b(\delta^b)-\frac{\partial\lambda^b}{\partial\delta}(\delta^b)\\
        e^{-\gamma(s-\delta^b-r^b(s,q,t))}&=1-\gamma\frac{\lambda^b(\delta^b)}{\frac{\partial\lambda^b}{\partial\delta}(\delta^b)}\\
        -\gamma(s-\delta^b-r^b(s,q,t))&=\log\left(1-\gamma\frac{\lambda^b(\delta^b)}{\frac{\partial\lambda^b}{\partial\delta}(\delta^b)}\right)\\
        s-\delta^b-r^b(s,q,t)&=-\frac{1}{\gamma}\log\left(1-\gamma\frac{\lambda^b(\delta^b)}{\frac{\partial\lambda^b}{\partial\delta}(\delta^b)}\right)\\
        s-r^b(s,q,t)&=\delta^b-\frac{1}{\gamma}\log\left(1-\gamma\frac{\lambda^b(\delta^b)}{\frac{\partial\lambda^b}{\partial\delta}(\delta^b)}\right)
    \end{align*}
    while a similar procedure for $\delta^a$ yields:
    \begin{align*}
        \frac{\partial}{\partial\delta}\left[\frac{\lambda^a(\delta)}{\gamma}(1-e^{-\gamma(s+\delta-r^a(s,q,t))})\right](\delta^a)&=0\\
        \frac{1}{\gamma}\left[\frac{\partial\lambda^a}{\partial\delta}(\delta^a)-\frac{\partial}{\partial\delta}\lambda^a(\delta^a)e^{-\gamma(s+\delta^a-r^a(s,q,t))}\right]&=0\\
        \frac{\partial\lambda^a}{\partial\delta}(\delta^a)-\frac{\partial\lambda^a}{\partial\delta}(\delta^a)e^{-\gamma(s+\delta^a-r^a(s,q,t))}+\gamma\lambda^a(\delta^a)e^{-\gamma(s+\delta^a-r^a(s,q,t))}&=0\\
        \left(\gamma\lambda^a(\delta^a)-\frac{\partial\lambda^a}{\partial\delta}(\delta^a)\right)e^{-\gamma(s+\delta^a-r^a(s,q,t))}&=-\frac{\partial\lambda^a}{\partial\delta}(\delta^a)\\
        -\left(\frac{\partial\lambda^a}{\partial\delta}(\delta^a)\right)e^{\gamma(s+\delta^a-r^a(s,q,t))}&=\gamma\lambda^a(\delta^a)-\frac{\partial\lambda^a}{\partial\delta}(\delta^a)\\
        e^{\gamma(s+\delta^a-r^a(s,q,t))}&=1-\gamma\frac{\lambda^a(\delta^a)}{\frac{\partial\lambda^a}{\partial\delta}(\delta^a)}\\
        \gamma(s+\delta^a-r^a(s,q,t))&=\log\left(1-\gamma\frac{\lambda^a(\delta^a)}{\frac{\partial\lambda^a}{\partial\delta}(\delta^a)}\right)\\
        s+\delta^a-r^a(s,q,t)&=\frac{1}{\gamma}\log\left(1-\gamma\frac{\lambda^a(\delta^a)}{\frac{\partial\lambda^a}{\partial\delta}(\delta^a)}\right)\\
        r^a(s,q,t)-s&=\delta^a-\frac{1}{\gamma}\log\left(1-\gamma\frac{\lambda^a(\delta^a)}{\frac{\partial\lambda^a}{\partial\delta}(\delta^a)}\right).
    \end{align*}
\end{proof}

To briefly summarise, the optimal bid and ask spreads $\delta^a$ and $\delta^a$ are
derived in an intuitive two-step procedure. First, we solve the HJB PDE (\ref{eq:3.19})
to obtain the reservation bid and ask prices. Then, we use the relations (\ref{eq:3.22})
and (\ref{eq:3.23}) to find the optimal bid and ask spreads $\delta^b(s,q,t)$ and 
$\delta^a(s,q,t)$ between the mid-price and obtimal bid and ask quotes respectively.

\section{Asymptotic Expansion in q}\label{sec:3.8}

The main computational difficulty in this advance is solving equation \ref{eq:3.19}
due to the order-arrival terms (the terms to be maximised) being highly nonlinear and
dependent on the inventory variable $q$. To get around this, \cite{AS2008} suggest
an asymptotic expansion of $\theta$ in the inventory variable $q$, and a linear 
approximation of the order arrival terms. 

Following on from our work in section \ref{sec:3.6}, we will assume that our arrival 
rates are symmetric and exponential given by 
\begin{equation}\label{eq:3.24}
    \lambda^a(\delta)+\lambda^b(\delta)=Ae^{-k\delta}
\end{equation}
in which case, our agents indifference prices $r^a(s,q,t)$ and $r^b(s,q,t)$
coincide with their frozen inventory values given in section \ref{sec:3.3}.

With some elementary calculus we can see that our exponential arrival rates satisfy
the following property:
\begin{equation*}
    \frac{\lambda(\delta)}{\frac{\partial\lambda}{\partial\delta}(\delta)}=\frac{Ae^{-k\delta}}{-kAe^{-k\delta}}=-\frac{1}{k}.
\end{equation*}
Hence by plugging in the relations (\ref{eq:3.22}) and (\ref{eq:3.23}) into the 
maximised terms in the HJB equation (\ref{eq:3.19}) under the assumption of symmetric
exponential arrival rates (3.24), we see that
\begin{align*}
    &\max\limits_{\delta^b}\left[\frac{\lambda^b(\delta^b)}{\gamma}(1-e^{\gamma(s-\delta^b-r^b)})\right]+\max\limits_{\delta^a}\left[\frac{\lambda^a(\delta^a)}{\gamma}(1-e^{-\gamma(s+\delta^a-r^a)})\right]\\
    &=\frac{Ae^{-k\delta^b}}{\gamma}\left(1-e^{\gamma\left(-\frac{1}{\gamma}\log\left(1+\frac{\gamma}{k}\right)\right)}\right)+\frac{Ae^{-k\delta^a}}{\gamma}\left(1-e^{-\gamma\left(\frac{1}{\gamma}\log\left(1+\frac{\gamma}{k}\right)\right)}\right)\\
    &=\left[\frac{A}{\gamma}\left(1-e^{-\log\left(1+\frac{\gamma}{k}\right)}\right)\right](e^{-k\delta^b}+e^{-k\delta^a})\\
    &=\left[\frac{A}{\gamma}\left(1-\frac{1}{1+\frac{\gamma}{k}}\right)\right](e^{-k\delta^b}+e^{-k\delta^a})\\
    &=\left(\frac{A}{\gamma}-\frac{A}{\gamma+\frac{\gamma^2}{k}}\right)(e^{-k\delta^b}+e^{-k\delta^a})\\
    &=\left(\frac{A\left(1+\frac{\gamma}{k}\right)-A}{\gamma+\frac{\gamma^2}{k}}\right)(e^{-k\delta^b}+e^{-k\delta^a})\\
    &=\left(\frac{A\frac{\gamma}{k}}{\gamma+\frac{\gamma^2}{k}}\right)(e^{-k\delta^b}+e^{-k\delta^a})\\
    &=\frac{A}{k+\gamma}(e^{-k\delta^b}+e^{-k\delta^a})
\end{align*}
which results in the simplified HJB equation below:
\begin{equation}\label{eq:simp-hjb}
    \begin{cases} 
        \theta_t+\frac{1}{2}\sigma^2\theta_{ss}-\frac{1}{2}\sigma^2\gamma\theta^2_s+\frac{A}{k+\gamma}(e^{-k\delta^a}+e^{-k\delta^b})=0\\
        \theta(s,q,T)=qs.
    \end{cases}
\end{equation}

Next, we consider an asymptotic expansion of $\theta$ in the inventory variable $q$:
\begin{equation}\label{eq:asymp}
    \theta(q,s,t)=\theta^0(s,t)+q\theta^1(s,t)+\frac{1}{2}q^2\theta^2(s,t)+...
\end{equation}
where the superscripts denote different functions, not powers - we do not use subscripts
to avoid conflicts with our notation for partial derivatives later on.

The exact relations for the reserve bid and ask prices obtained in lemma \ref{lem:3.7.3}
yield
\begin{gather}
    r^b(s,q,t)=\theta^1(s,t)+(1+2q)\theta^2(s,t)+...\\
    r^a(s,q,t)=\theta^1(s,t)+(-1-2q)\theta^2(s,t)+...\;.
\end{gather}
Then our reservation price
\begin{equation}\label{eq:reserve-price}
    r(s,q,t)=\frac{r^a(s,q,t)+r^b(s,q,t)}{2}=\theta^1(s,t)+2q\theta^2(s,t)
\end{equation}
follows immediately. We can interpret this expression nicely: $\theta^1$ is the reserve
price when the inventory is 0, and $\theta^2$ is our agent's sensitivity to changes in 
inventory. We might then expect that $\theta^2 < 0\;\forall(s,t)$, since then a long 
position will result in lower quotes (more willing to sell) and vice-versa. We also 
have that
\begin{equation}\label{eq:spread}
    \begin{aligned}
        \delta^a+\delta^b&=\frac{1}{\gamma}\log\left(1+\frac{\gamma}{k}\right)+r^a(s,q,t)-s+\frac{1}{\gamma}\log\left(1+\frac{\gamma}{k}\right)+s-r^b(s,q,t)\\
        &=r^a(s,q,t)-r^b(s,q,t)+\frac{2}{\gamma}\log\left(1+\frac{\gamma}{k}\right)\\
        &=-2\theta^2(s,t)+\frac{2}{\gamma}\log\left(1+\frac{\gamma}{k}\right)
    \end{aligned}
\end{equation}
through our approximation and the relations (\ref{eq:3.22}) and (\ref{eq:3.23}).
Now consider a first-order approximation of the order arrival term:
\begin{equation}\label{eq:approx-arrival}
    \frac{A}{k+\gamma}(e^{-\gamma\delta^a}+e^{-\gamma\delta^b})=\frac{A}{k+\gamma}(2-k(\delta^a+\delta^b)+...)
\end{equation}
where we notice that the linear term does not depend on the inventory $q$. Therefore,
by substituting (\ref{eq:asymp}) and (\ref{eq:approx-arrival}) into (\ref{eq:simp-hjb}) 
and grouping terms of order $q$ we obtain
\begin{equation}\label{eq:3.32}
    \begin{cases} 
        \theta^1_t+\frac{1}{2}\sigma^2\theta^1_{ss}=0\\
        \theta^1(s,T)=s.
    \end{cases}
\end{equation}
which admits the solution $\theta^1(s,t)=s$ by inspection. 

Grouping terms of order $q^2$ yields
\begin{equation}
    \begin{cases}
        \theta^2_t+\frac{1}{2}\sigma^2\theta^2_{ss}-\frac{1}{2}\sigma^2\gamma(\theta^1_s)^2=0\\
        \theta^2(s,T)=0
    \end{cases}
\end{equation}
which simplifies by our previous solution to (\ref{eq:3.32}) to
\begin{equation*}
    \begin{cases}
        \theta^2_t+\frac{1}{2}\sigma^2\theta^2_{ss}-\frac{1}{2}\sigma^2\gamma=0\\
        \theta^2(s,T)=0
    \end{cases}
\end{equation*}
with solution $\theta^2(s,t)=-\frac{1}{2}\sigma^2\gamma(T-t)$ by inspection.

Thus for this linear approximation of the order arrival term, we can substitute our
solutions back into (\ref{eq:reserve-price}) to obtain the same indifference price
\begin{equation}\label{eq:3.34}
    r(s,t)=s-q\gamma\sigma^2(T-t)
\end{equation}
as in the case where no trading is allowed. We quote a bid-ask spread that is symmetric
about this reservation price and is given by the below expression, which is again acquired
through subsituting our solutions for $\theta^1$ and $\theta^2$ back into (\ref{eq:spread}).
\begin{equation}\label{eq:3.35}
    \delta^a+\delta^b=\gamma\sigma^2(T-t)+\frac{2}{\gamma}\log\left(1+\frac{\gamma}{k}\right)
\end{equation}

\section{Summary}\label{sec:3.9}

In the above section, we have seen how the problem of finding the optimal behaviour
of a dealer in a limit orderbook, which we introduced in chapter \ref{chap:1}, can be
formulated as a stochastic control problem using the theoretical framework we built up
in chapter \ref{chap:2}. We have also noted a possible miscalculation or typographical
error in the short section of the original paper of \cite{AS2008} on the infinite-horizon
agent, which we address in section \ref{sec:3.4}. 

We then provide full derivations of all of the results presented in \cite{AS2008},
explain the use of the ansatz (\ref{eq:3.18}) as a key simplifying assumption, and walk
through the asymptotic approximation used to yield the final results that Avellaneda
and Stoikov present. 

In the next chapter, we will provide code and results for the
numerical simulation results that Avellaneda and Stoikov describe, attaining results 
that are very close to those presented in the original paper, and demonstrating some
of the concepts that we have worked with mathematically through visualisations of 
our simulations.