\section{Introduction}
In this chapter we introduce the idea of a stochastic control problem in one dimension, 
and construct a theoretical framework for the resolution of a regular solution, 
provided that such a solution exists (which is not guarunteed). We Primarily 
follow the text of \cite{Pham}. In section \ref{sec:2.2} we introduce the notion
of a controlled diffusion process and its solution. In section \ref{sec:2.3} we
consider a stochastic control problem over a finite time horizon, before introducing
the dynamic programming principle and Hamilton-Jacobi-Bellman equation in section \ref{sec:2.4}
and section \ref{sec:2.5} respectively. Finally, in section \ref{sec:2.6}, we consider
the verification theorem which allows us to validate the optimality of a candidate solution.
We then procede in \ref{sec:2.7} to put these tools to use through a worked example
in a financial context, setting us up to tackle the Avellaneda-Stoikov model in Chapter \ref{chap:3}.

\section{Controlled Diffusion Processes}\label{sec:2.2}

In the previous chapter we have considered It\^{o} processes that are governed either by constants or by
functions of time and/or state. Using this, we could for example model a stock price, the movement of a particle,
or any other system with the kinds of properties that we study above. If we have a portfolio of cash 
and an asset, we can model our wealth through time as a stochastic differential equation governed by the 
risk-free rate at which we earn returns on our cash, and the random fluctuations of the stock price. 

For the market-maker however, this is insufficient. We described intuitively at the beginning of chapter \ref{chap:1}
how a market maker might be able to influence the flow of orders they recieve over time, and hence their 
cash flow over time, by adjusting the limit bid and ask quotes that they send to the market. Hence,
the market makers portfolio value is governed by not only the fluctuations of the stock and the risk-free
rate, but also (stochastically) by the spread that they set. We thus need a model that allows our diffusion
process to be governed by not only functions of time and state, but also of some other process which we 
will call $\alpha$.

Throughout this chapter we will assume the background of a standard continuous and filtered probability space
$(\Omega,\mathcal{F},(\mathcal{F}_t)_{t\geq0},\mathbb{P})$ as defined above in chapter \ref{chap:1}.

\begin{definition}[Controlled Diffusion Process]
    We consider a control model where the state of the system is governed by 
    an $\mathbb{R}$-valued SDE:
    \begin{equation}\label{eq:2.1}
        \mathrm dX_t=b(t,X_t,\alpha_t)\mathrm ds+\sigma(t,X_t,\alpha_t)\mathrm dW_t
    \end{equation}
    where $W$ is a standard Wiener process. The control $\alpha=(\alpha_t)$ is a 
    \emph{progressively measurable} process valued in $A\subseteq\mathbb{R}^m$.
\end{definition}

The functions $b:\mathbb{R}^+\times\mathbb{R}\times A\rightarrow\mathbb{R}$
and $\sigma:\mathbb{R}^+\times\mathbb{R}\times A\rightarrow\mathbb{R}$ are measurable
in all of their arguments and satisfy a uniform Lipschitz condition in A:
There exists a $K\geq0$ such that $\forall\;x,y\in\mathbb{R}, \forall\;a\in A,$
\begin{equation}\label{eq:2.2}
    |b(x,a)-b(y,a)|+|\sigma(x,a)-\sigma(y,a)|\leq K|x-y|.
\end{equation}
In what follows, for $0\leq t\leq T<\infty$, we denote by $\mathcal{T}_{t,T}$ the
set of \emph{stopping times} valued in $[t,T]$.

\section{The Finite-Horizon Problem}\label{sec:2.3}

Fix a finite horizon $0<T<\infty$. We denote by $\mathcal{A}$ the set of control
processes $\alpha$ such that for any arbitrary $x\in\mathbb{R},$
\begin{equation}\label{eq:2.3}
    \mathbb{E}\left[\int_0^T|b(x,\alpha_t)|^2+|\sigma(x,\alpha_t)|^2\mathrm dt\right]<\infty.
\end{equation}
From Chapter \ref{chap:1}, conditions (\ref{eq:2.2}) and (\ref{eq:2.3}) ensure the existence
and uniqueness of a strong solution to the SDE (\ref{eq:2.1}) starting from any initial
condition $(t,x)\in[0,T]\times\mathbb{R}$ and with any control process $\alpha\in\mathcal{A}$.
We denote this unique strong solution with almost surely continuous sample paths by 
$\{X_s^{t,x},t\leq s\leq T\}$.\\
\textbf{Pham includes extra technical properties of the strong solution here which I may not require.}\\
Next we set out our functional objective. Let $f:[0,T]\times\mathbb{R}\times A\rightarrow\mathbb{R}$
and $g:\mathbb{R}\rightarrow\mathbb{R}$ be two measurable functions. We suppose that:

\begin{itemize}
    \item $g$ is lower-bounded \textbf{or}
    \item $g$ satisfies a quadratic growth condition: $|g(x)|\leq C(1+|x|^2)\;\forall\;x\in\mathbb{R}$ for some constant $C$ independent of $x$.
\end{itemize}

We also denote by $\mathcal{A}(t,x)$ the subset of controls $\alpha\in\mathcal{A}$ such that

\begin{equation}
    \mathbb{E}\left[\int_t^T|f(s,X_x^{t,x},\alpha_s)|\mathrm ds\right]<\infty
\end{equation}

for $(t,x)\in[0,T]\times\mathbb{R}$, and we assume that this set is not empty for all 
$(t,x)\in[0,T]\times\mathbb{R}$. We now define the \emph{gain function:}

\begin{definition}[Gain Function]
    \begin{equation}
        J(t,x,\alpha):=\mathbb{E}\left[\int_t^Tf(s,X_s^{t,x},\alpha_s)\mathrm ds+g(X_T^{t,x})\right]
    \end{equation}
    for all $(t,x)\in[0,T]\times\mathbb{R}$ and $\alpha\in\mathcal{A}(t,x).$
\end{definition}

Our objective is thus to maximise over possible control processes the gain function $J$,
and to do this we introduce the associated \emph{value function}:

\begin{definition}[Value Function]
    \begin{equation}\label{eq:2.6}
        v(t,x):=\sup_{\alpha\in\mathcal{A}(t,x)}J(t,x,\alpha).
    \end{equation}
\end{definition}

\begin{definition}[Optimal control]
    Given an initial condition $(t,x)\in[0,T]\times\mathbb{R}$, we say that $\hat{\alpha}\in\mathcal{A}(t,x)$
    is an optimal control if
    \begin{equation}
        v(t,x)=J(t,x,\hat{\alpha}).
    \end{equation}
\end{definition}

\begin{remark}
    A control process $\alpha$ of the form $\alpha_s=a(s,X_s^{t,x})$ for some measurable
    function $a:[0,T]\times\mathbb{R}\rightarrow A$ is called a \emph{Markovian} control.
\end{remark}

\textbf{Pham includes a remark here about constant controls and conditions for the equivalence of $\mathcal{A}$ and $\mathcal{A}(t,x)$ which I probably don't need.}\\
\textbf{Include words on the interpretation and intuition of the f, g, the gain and value functions}

\section{The Dynamic Programming Principle}\label{sec:2.4}

The Dynamic Programming Principle (DPP) is the fundamental tool upon which much of the 
theory of stochastic control relies. We formulate it as follows, considering only
the context of the finite-horizon problem described above.

\begin{theorem}[Dynamic Programming Principle]
    Let $(t,x)\in[0,T]\times\mathbb{R}.$ Then we have
    \begin{align}
        v(t,x)&=\sup_{\alpha\in\mathcal{A}(t,x)}\sup_{\theta\in\mathcal{T}_{t,T}}\mathbb{E}\left[\int_{t}^\theta f(s,X_s^{t,x},\alpha_s)\mathrm ds+v(\theta,X_\theta^{t,x})\right]\\
        &=\sup_{\alpha\in\mathcal{A}(t,x)}\inf_{\theta\in\mathcal{T}_{t,T}}\mathbb{E}\left[\int_{t}^\theta f(s,X_s^{t,x},\alpha_s)\mathrm ds+v(\theta,X_\theta^{t,x})\right]\\
    \end{align}
\end{theorem}
\begin{proof}[Proof of the DPP]
    By pathwise uniqueness of the SDE for X, for any admissible control $\alpha\in\mathcal{A}(t,x)$,
    for any $\theta\in\mathcal{T}_{t,T}$ and for all $s\geq\theta$
    \begin{equation}
        X_s^{t,x}=X_s^{\theta,X_\theta^{t,x}}.
    \end{equation}
    By the law of iterated expectations we then have
    \begin{align*}
        J(t,x,\alpha)&=\mathbb{E}\left[\int_t^Tf(s,X_s^{t,x},\alpha_s)\mathrm ds+g(X_T^{t,x})\right]\\
        &=\mathbb{E}\left[\mathbb{E}\left[\int_t^Tf(s,X_s^{t,x},\alpha_s)\mathrm ds+g(X_T^{t,x})|\mathcal{F}_\theta\right]\right]\\
        &=\mathbb{E}\left[\int_t^\theta f(s,X_s^{t,x},\alpha_s)\mathrm ds+\mathbb{E}\left[\int_\theta^Tf(s,X_s^{t,x},\alpha_s)\mathrm ds+g(X_T^{t,x})|\mathcal{F}_\theta\right]\right]\\
        &=\mathbb{E}\left[\int_t^\theta f(s,X_s^{t,x},\alpha_s)\mathrm ds+\mathbb{E}\left[\int_\theta^Tf(s,X_s^{t,x},\alpha_s)\mathrm ds+g(X_T^{t,x})\right]\right]\\
        &=\mathbb{E}\left[\int_t^\theta f(s,X_s^{t,x},\alpha_s)\mathrm ds+J(\theta,X_\theta^{t,x},\alpha)\right]
    \end{align*}
    and since $J(.,.,\alpha)\leq v$ and $\theta$ is arbitrary in $\mathcal{T}_{t,T}$
    we obtain
    \begin{align*}
        J(t,x,\alpha)&\leq\mathbb{E}\left[\int_t^\theta f(s,X_s^{t,x},\alpha_s)\mathrm ds+v(\theta,X_\theta^{t,x})\right]\\
        &\leq\inf_{\theta\in\mathcal{T}_{t,T}}\mathbb{E}\left[\int_t^\theta f(s,X_s^{t,x},\alpha_s)\mathrm ds+v(\theta,X_\theta^{t,x})\right]\\
        &\leq\sup_{\alpha\in\mathcal{A}(t,x)}\inf_{\theta\in\mathcal{T}_{t,T}}\mathbb{E}\left[\int_t^\theta f(s,X_s^{t,x},\alpha_s)\mathrm ds+v(\theta,X_\theta^{t,x})\right]
    \end{align*}
    and by taking the supremum over $\alpha$ in the left hand side, we obtain the second
    of the desired inequalities:
    \begin{equation}
        v(t,x)\leq\sup_{\alpha\in\mathcal{A}(t,x)}\inf_{\theta\in\mathcal{T}_{t,T}}\mathbb{E}\left[\int_t^\theta f(s,X_s^{t,x},\alpha_s)\mathrm ds+v(\theta,X_\theta^{t,x})\right].
    \end{equation}
    Next we fix an arbitrary control $\alpha\in\mathcal{A}(t,x)$ and $\theta\in\mathcal{T}_{t,T}.$
    By the definition of the value function, for any $\epsilon>0$ and $\omega\in\Omega$
    there exists an $\alpha^{\epsilon,\omega}\in\mathcal{A}(\theta(\omega),X_{\theta(\omega)}^{t,x}(\omega))$
    that is an $\epsilon$-optimal control for $v(\theta,X_{\theta(\omega)}^{t,x}(\omega))$, i.e.
    \begin{equation}\label{eq:2.13}
        v(\theta,X_{\theta(\omega)}^{t,x}(\omega))-\epsilon \leq J(\theta(\omega),X_{\theta(\omega)}^{t,x}(\omega),\alpha^{\epsilon,\omega}).
    \end{equation}
    \textbf{Maybe include some more intuition behind this point?}\\
    We now define the process
    \begin{equation}
        \hat{\alpha}_s(\omega)=\begin{cases}
            &\alpha_s(\omega), s\in[0,\theta(\omega)],\\
            &\alpha_s^{\epsilon,\omega}(\omega),s\in[\theta(\omega),T].
        \end{cases}
    \end{equation}
    It can be shown by the measurable selection theorem \\
    \textbf{do this in the appendix}\\
    that the process $\hat{\alpha}$ is progressively measurable, and so lies in $\mathcal{A}(t,x)$.
    Again by the law of iterated expectations and (\ref{eq:2.13}) we get 
    \begin{align*}
        v(t,x)\geq J(t,x,\hat{\alpha})=\mathbb{E}\left[\int_t^\theta f(s,X_s^{t,x},\alpha_s)\mathrm ds+J(\theta,X_\theta^{t,x},\alpha^\epsilon)\right]\\
        \geq\mathbb{E}\left[\int_t^\theta f(s,X_s^{t,x},\alpha_s)\mathrm ds+v(\theta,X_\theta^{t,x})\right]-\epsilon.
    \end{align*}
    Finally, by the fact that $\alpha\in\mathcal{A}(t,x),\theta\in\mathcal{T}_{t,T}$
    and $\epsilon>0$ are all arbitrary, we obtain the first inequality:
    \begin{equation}
        v(t,x)=\sup_{\alpha\in\mathcal{A}(t,x)}\sup_{\theta\in\mathcal{T}_{t,T}}\mathbb{E}\left[\int_{t}^\theta f(s,X_s^{t,x},\alpha_s)\mathrm ds+v(\theta,X_\theta^{t,x})\right].
    \end{equation}
\end{proof}

\begin{remark}[Equivalent Formulations]
    We normally write the DPP as 
    \begin{equation}\label{eq:2.16}
        v(t,x)=\sup_{\alpha\in\mathcal{A}(t,x)}\mathbb{E}\left[\int_{t}^\theta f(s,X_s^{t,x},\alpha_s)\mathrm ds+v(\theta,X_\theta^{t,x})\right],
    \end{equation}
    however it is sometimes useful to use the following equivalent formulation of the DPP:\\
    (i) For all $\alpha\in\mathcal{A}(t,x)$ and $\theta\in\mathcal{T}_{t,T}$:
    \begin{equation}\label{eq:2.17}
        v(t,x)\geq\mathbb{E}\left[\int_{t}^\theta f(s,X_s^{t,x},\alpha_s)\mathrm ds+v(\theta,X_\theta^{t,x})\right].
    \end{equation}
    (ii) For all $\epsilon>0,$ there exists $\alpha\in\mathcal{A}(t,x)$ such that for all $\theta\in\mathcal{T}_{t,T}$:
    \begin{equation}
        v(t,x)-\epsilon\leq\mathbb{E}\left[\int_{t}^\theta f(s,X_s^{t,x},\alpha_s)\mathrm ds+v(\theta,X_\theta^{t,x})\right].
    \end{equation}
    
\end{remark}

\textbf{Include words on the interpretation and intuition of the DPP.}

\section{Hamilton-Jacobi-Bellman Equation}\label{sec:2.5}

The Dynamic Programming Principle tells us that we can consider a stochastic control
problem as a sequence of smaller sub-problems defined over intervals of $[0,T]$
characterised by stopping times, i.e., $[0,T]=[0,\theta_1]\cup(\theta_1,\theta_2]\cup\dots\cup(\theta_{n},T]$
where $\theta_1\leq\dots\leq\theta_n\in\mathcal{T}_{t,T}.$ Thus, a natural thing to consider
is the following: What happens as $n\rightarrow\infty$ and correspondingly $\theta_{i+1}-\theta_i\rightarrow0$?
What we obtain is the Hamilton-Jacobi-Bellman equation (HJB) which describes the 
dynamics of the value function over small increments of time. In this chapter and what 
follows, we will use the HJB equation as follows:
\begin{itemize}
    \item Provide a formal derivation of the HJB equation.
    \item Obtain or try to show the existence of a smooth solution.
    \item Verification step: Show that the smooth solution is the value function.
    \item As a byproduct, we obtain an optimal feedback control.
\end{itemize}

\begin{theorem}[Hamilton-Jacobi-Bellman Equation]
    The dynamics of the value function $v(t,x)$ satisfy the following non-linear
    second-order partial differential equation:
    \begin{equation}
        \left\{
        \begin{aligned}
            &\frac{\partial v}{\partial t}(t,x)+\sup_{a\in A}\left[\mathcal{L}^av(t,x)+f(t,x,a)\right]=0\;\forall\;(t,x)\in[0,T)\times\mathbb{R}\\
            &v(T,x)=g(x)\;\forall\;x\in\mathbb{R}.
        \end{aligned}
        \right.
    \end{equation}
    where $\mathcal{L}^a$ is the operator associated to the diffusion (\ref{eq:2.1})
    and defined by (see Section \ref{sec:1.6})
    \begin{equation}
        \mathcal{L}^av=b(t,x,a)v_x+\frac{1}{2}\sigma(t,x,a)^2v_{xx}.
    \end{equation}
\end{theorem}
\begin{proof}
    Let us consider time $\theta=t+h$ and a constant control $\alpha_s=a$ for some arbitrary
    $a\in A$, in our slightly stronger variant of the DPP (\ref{eq:2.17}):
    \begin{equation}\label{eq:2.21}
        v(t,x)\geq\mathbb{E}\left[\int_{t}^{t+h} f(s,X_s^{t,x},a)\mathrm ds+v(t+h,X_{t+h}^{t,x})\right].
    \end{equation}
    By assuming that $v$ is smooth enough, we can apply It\^{o}'s formula between
    $t$ and $t+h$:
    \begin{equation}
        v(t+h,X_{t+h}^{t,x})=v(t,x)+\int_t^{t+h}\left(\frac{\partial v}{\partial t}+\mathcal{L}^av\right)(s,X_s^{t,x})\mathrm ds +\textrm{(local) martingale}.
    \end{equation}
    We can then substitute back into (\ref{eq:2.21}) to obtain
    \begin{equation}
        0\geq\mathbb{E}\left[\int_t^{t+h}\left(\frac{\partial v}{\partial t}+\mathcal{L}^av\right)(s,X_s^{t,x})+f(s,X_s^{t,x},a)\mathrm ds\right]
    \end{equation}
    which if we divide by $h$ and send $h\rightarrow0$ we yield
    \begin{equation}
        0\geq\frac{\partial v}{\partial t}(t,x)+\mathcal{L}^av(t,x)+f(t,x,a)
    \end{equation}
    by the mean-value theorem. Since this holds true for any $a\in A$, we obtain the
    inequality
    \begin{equation}\label{eq:2.25}
        -\frac{\partial v}{\partial t}(t,x)-\sup_{a\in A}[\mathcal{L}^av(t,x)+f(t,x,a)]\geq0.
    \end{equation}
    On the other hand, suppose that $\alpha^*$ is an optimal control. Then in (\ref{eq:2.16})
    we have
    \begin{equation}\label{eq:2.26}
        v(t,x)=\mathbb{E}\left[\int_t^{t+h}f(s,X_s^*,\alpha_s^*)\mathrm ds+v(t+h,X_{t+h}^*)\right],
    \end{equation}
    where $X^*$ is the solution to (\ref{eq:2.1}) starting from state $x$ at time $t$
    with control $\alpha^*.$ Again by It\^{o}'s formula we have that 
    \begin{equation}
        v(t+h,X^*_{t+h})=v(t,x)+\int_t^{t+h}\left(\frac{\partial v}{\partial t}+\mathcal{L}^av\right)(s,X_s^*)\mathrm ds +\textrm{(local) martingale}
    \end{equation}
    which we can again substitute back into (\ref{eq:2.26}) to obtain
    \begin{equation}
        0=\mathbb{E}\left[\int_t^{t+h}\left(\frac{\partial v}{\partial t}+\mathcal{L}^av\right)(s,X_s^*)+f(s,X_s*,a)\mathrm ds\right]
    \end{equation}
    and hence once again we divide by $h$ and send $h\rightarrow0$ yielding
    \begin{equation}
        -\frac{\partial v}{\partial t}(t,x)-\mathcal{L}^{\alpha^*_t}v(t,x)-f(t,x,\alpha_t^*)=0.
    \end{equation}
    Combining this with (\ref{eq:2.25}), $v$ should satisfy
    \begin{equation}
        -\frac{\partial v}{\partial t}(t,x)-\sup_{a\in A}\left[\mathcal{L}^av(t,x)+f(t,x,a)\right]=0\;\forall\;(t,x)\in[0,T)\times\mathbb{R},
    \end{equation}
    if the above supremum in $a$ is finite. This may arise when the control space $A$
    is unbounded, and we will see how to deal with this later on. We can also obtain
    the terminal condition associated to this PDE:
    \begin{equation}
        v(T,x)=g(x)\;\forall\;x\in\mathbb{R}
    \end{equation}
    which results immediately from the definition in (\ref{eq:2.6}) of the value function
    considered at the horizon time $T$.
\end{proof}

\section{Verification Theorem}\label{sec:2.6}

\begin{theorem}[Verification Theorem]
    Let $w:[0,T]\times\mathbb{R}\rightarrow\mathbb{R}$ be a continuous function which
    is continuously differentiable at least once in its first argument and twice
    in its second. Let $w$ also satisfy a quadratic growth condition, i.e. there
    exists a constant $C$ such that
    \begin{equation*}
        |w(t,x)|\leq C(1+x^2)\;\forall\;(t,x)\in[0,T]\times\mathbb{R}.
    \end{equation*}
    (i) Suppose that
    \begin{align}
        \frac{\partial w}{\partial t}(t,x)+\sup_{a\in A}[\mathcal{L}^aw(t,x)+f(t,x,a)]&\geq0,\;(t,x)\in[0,T]\times\mathbb{R}\\
        w(T,x)&\geq g(x),\;x\in\mathbb{R}.
    \end{align}
    Then $w\geq v$ on $[0,T]\times\mathbb{R}.$\\
    (ii) Suppose further that $w(T,.)=g$ and that there exists a measurable function
    $\hat{\alpha}:[0,T]\times\mathbb{R}\rightarrow A$ such that
    \begin{equation*}
        \frac{\partial w}{\partial t}(t,x)+\sup_{a\in A}[\mathcal{L}^aw(t,x)+f(t,x,a)]=\frac{\partial w}{\partial t}(t,x)+\mathcal{L}^{\hat{\alpha}(t,x)}w(t,x)+f(t,x,\hat{\alpha}(t,x))=0,
    \end{equation*}
    the SDE
    \begin{equation*}
        \mathrm dX_t=b(s,X_s,\hat{\alpha}(s,X_s))\mathrm ds+\sigma(s,X_s,\hat{\alpha}(s,X_s))\mathrm dW_s
    \end{equation*}
    admits a unique solution denoted by $\hat{X}_s^{t,x}$ given an initial condition
    $X_t=x$, and the process $\{\hat{\alpha}(s,\hat{X}_s^{t,x}):t\leq s\leq T\}$
    lies in $\mathcal{A}(t,x)$. Then
    \begin{equation}
        w=v\textrm{ on }[0,T]\times\mathbb{R}
    \end{equation}
    and $\hat{\alpha}$ is an optimal Markovian control.
\end{theorem}
\begin{proof}
    
\end{proof}

\section{A Worked Example}\label{sec:2.7}

\subsection*{Finite-Horizon Merton Portfolio Allocation Problem}