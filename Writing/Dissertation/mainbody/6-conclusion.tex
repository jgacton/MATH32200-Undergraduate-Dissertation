%\begin{itemize}
%    \item C1 Measure Theory 
%    \item C1 SDEs and It\^{o}'s lemma
%    \item C1 Controlled diffusion processes
%    \item C2 Dynamic Programming
%    \item C2 Hamilton-Jacobi-Bellman
%    \item C3 Market-making
%    \item C3 The avellaneda-stoikov model
%    \item C4 Numerical simulation and results
%\end{itemize}

% Clearly state the answer to your main research question

The main goal of this report was to clearly formulate and solve the problem of 
market-making under inventory risk in the framework of stochastic optimal control.
With the background material in stochastic calculus and stochastic differential 
equations from chapter \ref{chap:1}, and our development of the theory of 
stochastic optimal control in chapter \ref{chap:2}, we were then able to formulate 
and prove the results of \cite{AS2008} concerning market-making under inventory 
risk in chapter \ref{chap:3}. Indeed, we may even have corrected an error in their calculation
of reservation prices in the infinite-horizon case, although this does not impact the 
main result of the paper. We go on to perform the same numerical simulations as 
in \cite{AS2008}, with a larger sample size, and replicate their results.

Our conclusion is thus that under some potentially quite restrictive assumptions,
we can formulate a general problem statement for the dealer operating in a limit 
orderbook in terms of a value function to be optimised. We can then derive the 
Hamilton-Jacobi-Bellman equation, which we may solve explicitly (if a solution exists),
numerically (if this is computationally tractable), or in our case, approximately, using 
an important ansatz and analytical approximations.

% Summarise and reflect on your research process
% Make recommendations for future work on your topic

As mentioned previously, the assumptions under which we can formulate the general 
problem of a dealer may not be the most realistic. On the finite time-horizon which
we consider here, modelling the asset as a Brownian motion without drift may be valid,
but if building a market-making model for an asset which is continuously traded 24/7
such as currencies or cryptocurrencies, this may be invalid as over longer time horizons
assets tent to exhibit drift (equities), mean-reversion (currencies), and fatter tails 
to the distribution of returns than a normal or log-normal distribution would suggest.

We also ignore transaction costs. Depending on the exact rules of the exchange running 
the orderbook, market-makers may have to pay a small percentage fee every time their 
limit orders are executed. On the other hand, some exchanges give rebates to market-
makers, paying them a small percentage of their trading volume in exchange for the 
liquidity the market-maker provides to other participants on the exchange.

Another important factor is the fundamental physical constraints of market-making,
particularly in developed, electronic markets with high trading volumes and lots of
participants. Trades tend to be executed on millisecond timescales, so latency becomes
an important factor, and many firms in this space have developed custom hardware (FPGAs)
which they pay the exchange to plug directly into the exchange's servers. Simply put,
there is a fundamental limit to the speed at which you can update your quotes to ensure 
you are always at the head of the queue when market orders reach the market, and if 
another market-maker is closer to that fundamental limit than you, then you won't be 
able to trade optimally. 

The constant arrival frequency of market orders is another potentially unrealistic 
assumption. Trading volume tends to peak around the opening and closing bell of the 
trading session as new information that may have revealed pre or post-market is traded
on, before settling down for the rest of the day. Corporate events such as earnings 
releases or macroeconomic events such as central bank interest rate decisions also 
cause short-term spikes in trading volume. To mitigate this,

% Show what new knowledge you have contributed to your field

In chapter \ref{chap:1} we introduced the motivation for the market-making problem,
that to ensure smooth operation and liquidity in quote driven markets, a dealer can 
continuously quote bid and ask prices in a security in return for profiting the bid-
ask spread mutliplied across a large volume of transactions. We then discussed how 
financial markets are organised in the limit orderbook, which we illustrated in 
\ref{fig:orderbook}. We presented the ideas that trading determines prices, and that 
the dealer is incentivised to minimise their inventory both to avoid taking on 
unecessary market risk and asymmetric information risk.

Before moving on to discuss stochastic control, we first needed to acquire some 
prerequisite knowledge from probability theory and analysis, including a formal 
treatment of probability theory as a subset of measure theory, expectation as an
application of the lebesgue integral, and conditional expectation in a measure-theoretic
setting. We then recapped some basic definitions of stochastic processes in continuous
time, before turning to the concept of stochastic integration given by It\^{o}.

Next, we considered stochastic calculus in more depth, looking at It\^{o}'s lemma 
and the concept of a stochastic differential equation and its solution, and saw a brief 
example through the geometric Brownian motion.

Chapter \ref{chap:2} extended our understanding of SDEs to controlled SDEs where we 
have an auxiliary process $\alpha$ influencing the dynamics of our system. We then 
introduced the setup of finite horizon stochastic control, where we wish to maximise 
some function of our stochastic process over the space of possible control processes.

The crucial part of this theory is the Dynamic Programming Principle, which simply states
that our optimisation problem can be split up into an optimisation from time 0 to 
some stopping time $\theta$, and then from $\theta$ to $T$, and that this process of 
splitting can be repeated as many times as we would like. Continuing this line of reasoning
to infinitessimal intervals of time gave us the Hamilton-Jacobi-Bellman equation for 
our maximised gain function - the value function.

In chapter \ref{chap:3}, we returned to the problem of the dealer, now with a clearer 
idea of how we can formulate the behaviour of the dealer as a stochastic control problem:
maximising expected utility of final wealth over the space of possible bid and ask quotes
that the dealer can choose during $[0,T]$. However, we were faced with a potentially 
very complex HJB equation, and so sought to obtain an approximate solution through 
asymptotic approximations and an important simplifying ansatz. Utilising these tools 
gave us a simple approximate solution in terms of our pre-determined model parameters.

Chapter \ref{chap:5} then allowed us to demonstrate the performance of the derived 
strategy against that of a simple benchmark strategy, comparing mean and standard 
deviation of profit and final inventory over 10000 simulated trading sessions. We 
managed to replicate the results of \cite{AS2008}, and compare the performance of the 
strategy with different risk-aversion parameters $\gamma$. Finally, we illustrate 
the approach of the model in the plots \ref{fig:sample-paths}, \ref{fig:inventory},
and \ref{fig:pnl}.
