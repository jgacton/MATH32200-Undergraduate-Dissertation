The main goal of this project was to clearly formulate and solve the problem of 
market-making under inventory risk in the framework of stochastic optimal control.
With the background material in stochastic calculus and stochastic differential 
equations from chapter \ref{chap:1}, and our development of the theory of 
stochastic optimal control in chapter \ref{chap:2}, we were then able to formulate 
and prove the results of \textcite{AS2008} concerning market-making under inventory 
risk in chapter \ref{chap:3}. Indeed, we may even have corrected an error in their calculation
of reservation prices in the infinite-horizon case, although this does not impact the 
main result of their paper. We go on to perform the same numerical simulations as 
in \textcite{AS2008}, with a larger sample size, and replicate their results.

Our conclusion is thus that under some potentially quite restrictive assumptions,
we can formulate a general problem statement for the dealer operating in a limit 
orderbook in terms of a value function to be optimised. We can then derive the 
Hamilton-Jacobi-Bellman equation, which we may solve explicitly (if a solution exists),
numerically (if this is computationally tractable), or in our case, approximately, using 
an important ansatz and analytical approximations.

As mentioned above, the assumptions under which we can formulate the general 
problem of a dealer may not be the most realistic. One particular example of this 
is that we assume the arrival frequency of market orders to be constant. Trading volume 
tends to peak around the opening and closing bell of the trading session (
\cite{volume}) as new information that may have revealed pre or post-market is traded
on, before settling down for the rest of the day. Corporate events such as earnings 
releases also cause short-term spikes in trading volume (\cite{EarningsPremium}). 

To incorporate these exogenous events into our current model would require our Poisson 
intensity $\lambda$ to be a function of both distance to the midprice and time, which 
would introduce extra complexity and potentially make our problem harder to solve. 

However, a possible alternative model for incoming market orders would be the Hawkes process
(\cite{Hawkes}), a kind of self-exciting Poisson process where arrivals increase the 
probability of more arrivals in the near future. This may be a more natural model for 
market-order arrivals as trades beget more trades in reaction. \textcite{Hawkes2}
presents an application of the Hawkes process to model order arrivals. 

Therefore, we recommend that future research in this area attempts to incorporate 
such self-exciting processes into the current stochastic control framework.
