\thispagestyle{plain}
\mbox{}
\vspace{60mm}
\begin{center}
    \textbf{Abstract}
\end{center}
This project presents a review of the mathematical theory that attempts to model the 
dynamics of an automated market-maker under inventory risk in financial markets. We 
begin by outlining financial markets, their participants and their microstructure, 
before discussing the requisite mathematical tools from probability theory, stochastic 
analysis, stochastic calculus and stochastic control. Next, we investigate the seminal 
2008 paper ``High-Frequency Trading in a Limit Order Book'' by Avellaneda and Stoikov 
(2008), which formalises the approach of a market-maker trading through limit orders 
and utilises the dynamic programming principle to solve for the market-maker's optimal 
bid and ask quotes. We then present an extension to this model which extends the 
process governing the underlying stock price from a simple Brownian Motion to the more 
standard Geometric Brownian Motion. Finally, we implement the model in Python and 
present it's empirical results when back-tested on real order book data from a 
cryptocurrency exchange.
\newpage